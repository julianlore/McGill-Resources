~\\\color{Orange}
5. Creating new methods
\\ 2 types of methods, methods someone else wrote (library methods) or methods you wrote
\\ Writing own allows to group many cmds into 1
\\Inside a class
\\ \colorbox{Cyan}{Method header} is where you give names and stuff for a method like so:
\\ public static \colorbox{Blue}{int} \colorbox{Red}{anotherMethod}(\colorbox{OliveGreen}{double a})
\\ \textcolor{Blue}{int rep output/return type, void if none, var type otherwise. If not void, return blabla; will return to method that called method with value of blabla}
\\ return statement has to be reached, only have 1 return in an if, compiler error, because a return has to be able to be reached!
\\ 2 ifs, 1 return in each, not compile! 1 if, 1 else, 1 return in each, will compile, cuz return reached no matter what
\\ Only 1 return statement will be reached during an exec of a method, because method is left once it hits $1^{st}$ return
\\ \textcolor{Red}{method name}
\\ \textcolor{OliveGreen}{Input type, name it will rep inside this meth}
\\ New meth defs new cmd can use in program
\\ Some just do things: robot.move();
\\ Others give values: double x = Math.sqrt(40);
\\ if meth belongs to same class as meth calling, don't need to write class, as opposed to lib methods
\\ Loc of meth (before or after current meth), does not matter, will scan through whole prog
\\ \colorbox{Red}{Remember} declaring a var in a meth and trying to use in a diff method $\rightarrow$ compiler error!
\\ Advantages: code reusability, reduce code dupe, easier debug, problems decomposed, hides tricky logic, easier to read and understand
\\ Disadvantage: a little overhead to set up in beginning (not really)
\\ Modifying given values in another meth will not affect the vals of the meth calling it (unless return modified value and assign it)