\color{Sepia}
\section{Expressions}
Most basic are numbers, strings and booleans.
\\ 3 -> int
\\ +,-,*,/ are operators for ints
\\ 3.0 -> float
\\+.,-.,*.,/. are operators for floats
\\ This is because OCaml does not keep types during runtime.
\begin{lstlisting}
"comp302" -> string
'a' -> char
true -> bool
true || false -> bool
true && false -> bool
\end{lstlisting}
\colorbox{Yellow}{if statements}
\begin{lstlisting}
if 0 = 0 then 1.0
else 2.0
if 0 = 0 then 1.0
else 'a'
\end{lstlisting}
Second line won't work with typechecker, both parts of if statement
must eval to same thing.
\\ $3/0$ will pass typechecker but will throw exception at runtime if
it gets run.
\color{Blue}
\section{Binding}
\colorbox{Yellow}{variables}
\begin{lstlisting}
let <name> = <exp>
let x = 3 * 3
\end{lstlisting}
x is bound to 9. Values are bounded to names. It
\textcolor{red}{cannot} be updated or changed, only overshadowed, for
ex.
\begin{lstlisting}
let x = 4
\end{lstlisting}
When we look up x now, we look up the most recent x on the binding
stack. Garbage collector may remove previous one if there is nothing
else using it (can have a function that was defined before new x but
uses old x)
\colorbox{Yellow}{scope}
\begin{lstlisting}
let <name> = <expr1> in
  <expr2>
\end{lstlisting}
scope of expr1 ends after expr2
Functions are values. Func names bind name to body.
\\ Recursive functions declared using \textcolor{Green}{\texttt{let
    rec}}
\begin{lstlisting}
let rec fact n =
  if n = 0 then 1
  else n * fact (n-1)
\end{lstlisting}
\colorbox{Yellow}{Tail-recursive} functions are recursive funcs
that have nothing to do except return final val. i.e. no need to save
stack frame. Include a parameter to \textcolor{Green}{accumulate
  result}
\begin{lstlisting}
let rec fact_tr n =
  let rec f (n, m) = 
    if n = 0 then m
    else f(n-1, n*m)
    in f(n,1)
  \end{lstlisting}

\color{OliveGreen}
\section{Types}
Defining a type:
\begin{lstlisting}
type suit = Clubs |
  Spades | Hearts
  | Diamonds
\end{lstlisting}
Order declared does not matter. Clubs, Spades, etc are \textbf{constructors}
and must begin with a \textbf{capital letter}
\paragraph{Recursively defined}
Define hand inductively. Empty is of type hand. If c is a card and h
is a hand, then Hand(c, h) is a hand. Nothing else is a hand.
\begin{lstlisting}
type hand = Empty |
  Hand of card * hand
let hand0:hand = Empty
let hand1:hand =
  Hand((Ace, Hearts),
  Empty)
\end{lstlisting}
\colorbox{Yellow}{Mutual recursive data type}
\begin{lstlisting}
type 'a forest = Forest
  of ('a tree) list
and 'a tree = Empty |
  Node of 'a * 'a forest
\end{lstlisting}
\colorbox{Yellow}{Option Data Type}
\begin{lstlisting}
type 'a option = None
  | Some of 'a
\end{lstlisting}
Used in case a function might not return something.
\color{Cyan}
\section{Pattern Matching}
\begin{lstlisting}
match <exp> with
| <pattern> -> <exp>
| <pattern> -> <exp>
...
\end{lstlisting}
exp we're analyzing is called \textit{scrutinee}.
\color{Orange}
\section{Arguments}
Passing all args at same time: 'a * 'b -> 'c
\\ One arg at a time: 'a -> 'b -> 'c
\\curry: (('a * 'b -> 'c) -> 'a -> 'b -> 'c)
\begin{lstlisting}
let curry f =
  (fun x y -> f (x,y))
\end{lstlisting}
uncurry: (('a -> b' -> 'c) -> ('a * b' -> 'c))
\begin{lstlisting}
let uncurry f =
  (fun (x,y) -> f x y)
\end{lstlisting}
Note that function types are right associative: 'a -> 'b -> 'c = 'a ->
('b -> 'c) and function application is left associative: f 1 2 = (f 1) 2
\color{Magenta}
\section{Proofs}
$e \Downarrow v$: e evals to v in multiple steps (Big-Step).
\\$e\Rightarrow e'$: e evals in one step to e' (small-step
(single))
\\$e \implies^* e'$: e evals in multiples steps to e' (small-step
(multiple))
\colorbox{Yellow}{Structural induction}
\begin{lstlisting}
type 'a list =
  nil | :: of 'a *
        'a list
\end{lstlisting}
To inductively prove about lists, prove for empty list, assume it
holds for lists t and then show for lists h ::\ t.
\begin{lstlisting}
type 'a tree =
  Empty | Node of 'a *
         'a tree *
         'a tree
\end{lstlisting}
To inductively prove about trees, prove for empty tree, assume for
trees l and r and show for tree Node(a, l, r)
\\\colorbox{Yellow}{Inductive Proof}
\begin{lstlisting}
let rec r_app l1
  l2 = match l1 with
  | [] -> l2
  | x::xs -> r_app xs
  (x::l2)
  
  let r_app' l1
  l2 =
  let rec rev l =
  match l with |[]->[]
  |x::xs -> xs @ [x] in
  let rec app l1 l2
  = match l1 with
  |[]->l2
  |x::l1'->x::(app l1' l2)
  in app
  (rev l1) l2
\end{lstlisting}
For all lists l1 l2, r\_app l1 l2= r\_app' l1 l2
\\ Proof by structural induction on l1
\\ Base: l1 = [].
\\ r\_app l1 l2 = r\_app [] l2 = l2 (by rev\_app) = app []
l2 (by def of app) = app (rev []) l2 (def of rev) = app (rev l1) l2 =
rev\_app' l1 l2 (by def of rev\_app')
\\ Step: l1 = h ::\ t,
\\ IH: For all l2, rev\_app t l2 = rev\_app' t l2.
\\ rev\_app' l1 l2 = rev\_app (h ::\ t) l2 = rev\_app = t (h ::\ l2)
(by def of rev\_app) = rev\_app' t (h ::\ l2) (by IH) = app (rev t) (h
::\ (app [] l2)) (def app) = app (rev t) (app [h] l2) (def app) = app
(app (rev t) [h]) l2 (ass of app) = app (rev (h ::\ t) l2) (def
rev\_app') = rev\_app' l1 l2   
\color{Black}
\section{Higher Order Functions}
Used to abstract over common functionality.
\\Non-generic sum: int * int -> int
\\Generic sum with fun as arg: (int -> int) -> int * int -> int
\\ Common hofs:
\begin{lstlisting}
List.map: ('a -> 'b) ->
  'a list -> 'b list
List.filter: ('a
  -> bool)
  -> 'a list -> 'a list
(* Folds from l -> r *)
List.fold_right:
  ('a -> 'b -> 'b) ->
  'a list -> 'b -> 'b
(* Folds from r -> l *)
List.fold_left:
  ('a -> 'b -> 'a) ->
  'a -> 'b list ->'b
List.for_all:
  ('a -> bool) ->
  'a list -> bool
List.exists:
  ('a -> bool) ->
  'a list -> bool
\end{lstlisting}
Anonymous functions: (fun x -> x+1)
\\ (function -> |\_ |else), equiv to matching argument
\\ Implementing them: map -> apply fun to head and prepend. Return
empty for empty list. filter -> Prepend on tail called if true, else
call again on tail.
\\fold\_right f l b-> ret base if empty, else f h
(fold\_right f t b)
\\fold\_left f l b-> ret base if nil, else fold\_left on f t and (f h b), new
base is f h b
\color{Plum}
\section{Partial Evaluation}
\begin{lstlisting}
let plus x y = x + y
let plus3 = (plus 3)
let plus3' = (fun x ->
  plus x 3)
\end{lstlisting}
plus: int -> int -> int
\\plus3: int -> int, although it's really a fun y -> 3 + y
\\plus3' is really a fun x -> x + 3
\\ Partial evaluation doesn't evaluate inside the function, just plugs
in the value you gave it. If you want to force it to evaluate a
certain part, you must define the part to evaluate using let z = x in
(fun y -> z + y)
%%% Local Variables:
%%% mode: latex
%%% TeX-master: "Midterm"
%%% End:
  