% \color{Sepia}
% \section{Expressions}
% Most basic are numbers, strings and booleans.
% \\ 3 -> int
% \\ +,-,*,/ are operators for ints
% \\ 3.0 -> float
% \\+.,-.,*.,/. are operators for floats
% \\ This is because OCaml does not keep types during runtime.
% \begin{lstlisting}
% "comp302" -> string
% 'a' -> char
% true -> bool
% true || false -> bool
% true && false -> bool
% \end{lstlisting}
% \colorbox{Yellow}{if statements}
% \begin{lstlisting}
% if 0 = 0 then 1.0
% else 2.0
% if 0 = 0 then 1.0
% else 'a'
% \end{lstlisting}
% Second line won't work with typechecker, both parts of if statement
% must eval to same thing.
% \\ $3/0$ will pass typechecker but will throw exception at runtime if
% it gets run.
% \color{Blue}
% \section{Binding}
% \colorbox{Yellow}{variables}
% \begin{lstlisting}
% let <name> = <exp>
% let x = 3 * 3
% \end{lstlisting}
% x is bound to 9. Values are bounded to names. It
% \textcolor{red}{cannot} be updated or changed, only overshadowed, for
% ex.
% \begin{lstlisting}
% let x = 4
% \end{lstlisting}
% When we look up x now, we look up the most recent x on the binding
% stack. Garbage collector may remove previous one if there is nothing
% else using it (can have a function that was defined before new x but
% uses old x)
% \colorbox{Yellow}{scope}
% \begin{lstlisting}
% let <name> = <expr1> in
%   <expr2>
% \end{lstlisting}
% scope of expr1 ends after expr2
% Functions are values. Func names bind name to body.
% \\ Recursive functions declared using \textcolor{Green}{\texttt{let
%     rec}}
% \begin{lstlisting}
% let rec fact n =
%   if n = 0 then 1
%   else n * fact (n-1)
% \end{lstlisting}
% \colorbox{Yellow}{Tail-recursive} functions are recursive funcs
% that have nothing to do except return final val. i.e. no need to save
% stack frame. Include a parameter to \textcolor{Green}{accumulate
%   result}
% \begin{lstlisting}
% let rec fact_tr n =
%   let rec f (n, m) = 
%     if n = 0 then m
%     else f(n-1, n*m)
%     in f(n,1)
%   \end{lstlisting}

\color{OliveGreen}
\section{Types}
Defining a type:
\begin{lstlisting}
type suit = Clubs |
  Spades | Hearts
  | Diamonds
\end{lstlisting}
Order declared does not matter. Clubs, Spades, etc are \textbf{constructors}
and must begin with a \textbf{capital letter}
\vspace{-1 em}
\paragraph{Recursively defined}
Define hand inductively. Empty is of type hand. If c is a card and h
is a hand, then Hand(c, h) is a hand. Nothing else is a hand.
\begin{lstlisting}
type hand = Empty |
  Hand of card * hand
let hand0:hand = Empty
let hand1:hand =
  Hand((Ace, Hearts),
  Empty)
\end{lstlisting}
\colorbox{Yellow}{Mutual recursive data type}
\begin{lstlisting}
type 'a forest = Forest
  of ('a tree) list
and 'a tree = Empty |
  Node of 'a * 'a forest
\end{lstlisting}
\colorbox{Yellow}{Option Data Type}
\begin{lstlisting}
type 'a option = None
  | Some of 'a
\end{lstlisting}
Used in case a function might not return something.
\vspace{-1 em}
\paragraph{Types of following exprs}
\begin{itemize}
  \setlength\itemsep{-0.5 em}
\item \texttt{3+2}, type = int, val = 5, no effect
\item \texttt{55}, int, 55, no
\item \texttt{fun x -> x+3*2}, \texttt{int -> int}, \texttt{<fun>},
  no
\item \texttt{((fun x->match x with [] -> true | y::ys -> false), 3.2
    *. 2.0)}, \texttt{(a' list -> bool) * float},
  \texttt{(<fun>,6.4)}, no
\item \texttt{let x = ref 3 in x := !x + 2},  unit, (), updates val of
  x to 5
\item \texttt{fun x -> x := 3}, \texttt{ref 'a -> unit},
  \texttt{<fun>}, no
\item \texttt{fun x -> (x := 3; x)}, \texttt{ref 'a -> ref 'a}, fun,
  no
\item \texttt{fun x -> (x := 3; !x)}, \texttt{ref 'a -> 'a}, fun, no
\end{itemize}

\color{Cyan}
\section{Pattern Matching}
\begin{lstlisting}
match <exp> with
| <pattern> -> <exp>
| <pattern> -> <exp>
...
\end{lstlisting}
exp we're analyzing is called \textit{scrutinee}.
\color{Orange}
\section{Arguments}
Passing all args at same time: 'a * 'b -> 'c
\\ One arg at a time: 'a -> 'b -> 'c
\\curry: (('a * 'b -> 'c) -> 'a -> 'b -> 'c)
\begin{lstlisting}
let curry f =
  (fun x y -> f (x,y))
\end{lstlisting}
uncurry: (('a -> b' -> 'c) -> ('a * b' -> 'c))
\begin{lstlisting}
let uncurry f =
  (fun (x,y) -> f x y)
\end{lstlisting}
Note that function types are right associative: 'a -> 'b -> 'c = 'a ->
('b -> 'c) and function application is left associative: f 1 2 = (f 1) 2
\color{Magenta}
\section{Proofs}
$e \Downarrow v$: e evals to v in multiple steps (Big-Step).
\\$e\Rightarrow e'$: e evals in one step to e' (small-step
(single))
\\$e \implies^* e'$: e evals in multiples steps to e' (small-step
(multiple))
\colorbox{Yellow}{Structural induction}
\begin{lstlisting}
type 'a list =
  nil | :: of 'a *
        'a list
\end{lstlisting}
To inductively prove about lists, prove for empty list, assume it
holds for lists t and then show for lists h ::\ t.
\begin{lstlisting}
type 'a tree =
  Empty | Node of 'a *
         'a tree *
         'a tree
\end{lstlisting}
To inductively prove about trees, prove for empty tree, assume for
trees l and r and show for tree Node(a, l, r)
\\\colorbox{Yellow}{Inductive Proof}
\begin{lstlisting}
let rec r_app l1
  l2 = match l1 with
  | [] -> l2
  | x::xs -> r_app xs
  (x::l2)
  
  let r_app' l1
  l2 =
  let rec rev l =
  match l with |[]->[]
  |x::xs -> xs @ [x] in
  let rec app l1 l2
  = match l1 with
  |[]->l2
  |x::l1'->x::(app l1' l2)
  in app
  (rev l1) l2
\end{lstlisting}
For all lists l1 l2, r\_app l1 l2= r\_app' l1 l2
\\ Proof by structural induction on l1
\\ Base: l1 = [].
\\ r\_app l1 l2 = r\_app [] l2 = l2 (by rev\_app) = app []
l2 (by def of app) = app (rev []) l2 (def of rev) = app (rev l1) l2 =
rev\_app' l1 l2 (by def of rev\_app')
\\ Step: l1 = h ::\ t,
\\ IH: For all l2, rev\_app t l2 = rev\_app' t l2.
\\ rev\_app' l1 l2 = rev\_app (h ::\ t) l2 = rev\_app = t (h ::\ l2)
(by def of rev\_app) = rev\_app' t (h ::\ l2) (by IH) = app (rev t) (h
::\ (app [] l2)) (def app) = app (rev t) (app [h] l2) (def app) = app
(app (rev t) [h]) l2 (ass of app) = app (rev (h ::\ t) l2) (def
rev\_app') = rev\_app' l1 l2   
\color{Black}
\section{Higher Order Functions}
Used to abstract over common functionality.
\\Non-generic sum: int * int -> int
\\Generic sum with fun as arg: (int -> int) -> int * int -> int
\\ Common hofs:
\begin{lstlisting}
List.map: ('a -> 'b) ->
  'a list -> 'b list
List.filter: ('a
  -> bool)
  -> 'a list -> 'a list
(* Folds from l -> r *)
List.fold_right:
  ('a -> 'b -> 'b) ->
  'a list -> 'b -> 'b
(* Folds from r -> l *)
List.fold_left:
  ('a -> 'b -> 'a) ->
  'a -> 'b list ->'b
List.for_all:
  ('a -> bool) ->
  'a list -> bool
List.exists:
  ('a -> bool) ->
  'a list -> bool
\end{lstlisting}
Anonymous functions: (fun x -> x+1)
\\ (function -> |\_ |else), equiv to matching argument
\\ Implementing them: map -> apply fun to head and prepend. Return
empty for empty list. filter -> Prepend on tail called if true, else
call again on tail.
\\fold\_right f l b-> ret base if empty, else f h
(fold\_right f t b)
\\fold\_left f l b-> ret base if nil, else fold\_left on f t and (f h b), new
base is f h b
\color{Plum}
\section{Partial Evaluation}
\begin{lstlisting}
let plus x y = x + y
let plus3 = (plus 3)
let plus3' = (fun x ->
  plus x 3)
\end{lstlisting}
plus: int -> int -> int
\\plus3: int -> int, although it's really a fun y -> 3 + y
\\plus3' is really a fun x -> x + 3
\\ Partial evaluation doesn't evaluate inside the function, just plugs
in the value you gave it. If you want to force it to evaluate a
certain part, you must define the part to evaluate using let z = x in
(fun y -> z + y)
\color{Blue}
\section{Effects}
\texttt{let x = ref 0}, compare addr with \texttt{t == s}, compare
content with \texttt{t=s}, read value with \texttt{!x}, update val
\texttt{x := 3}, pattern match val with \texttt{let \{contents = x\} =
ref 0} gives \texttt{x = 0}. 
\begin{lstlisting}
type counter_object = {tick : unit -> int ; reset: unit -> unit}

let newCounter () =
 let counter = ref 0 in 
  {tick = (fun () -> counter := !counter + 1; !counter) ; 
   reset = fun () -> counter := 0}
\end{lstlisting}
\color{Brown}
\section{Exceptions}
Force to consider exceptional case, can segregate special case from
others (less clutter), \textbf{divert control flow}
\\ \texttt{3/0} has type int, but no val, has an \textbf{effect},
raises run-time exception \texttt{Division\_by\_zero}
\begin{lstlisting}
let head (x::t) = x in
  head []
\end{lstlisting}
Raises \texttt{Match\_failure}
\begin{lstlisting}
  (*Make a new ex* )
  exception Domain
  raise Domain
\end{lstlisting}
Backtrack through tree using exceptions
\begin{lstlisting}
let rec find_gen  t k = match t with
| Empty -> raise NotFound
| Node (l, (k',d), r) -> 
if k = k' then d else 
try find_gen l k with NotFound -> find_gen r k

let rec change coins amt = 
 if amt = 0 then [] else 
  begin match coins with 
   | [] -> raise Change
   | coin::cs ->
    if coin > amt then change cs amt
    else 
     try 
      coin :: change coins (amt - coin)
       with Change -> change cs amt
  end
\end{lstlisting}
\color{Emerald}
\section{Modules}
sig for defining signature, struct for actual implem
\begin{lstlisting}
 module type CURRENCY = 
sig 
 type t 
 val unit : t 
 val plus : t -> t -> t 
 val prod : float -> t -> t 
 val toString : t -> string
end;;
  
module Float = 
struct 
 type t = float 
 let unit = 1.0 
 let plus = (+.) 
 let prod = ( *. ) 
end;;

\end{lstlisting}
\texttt{module Euro = (Float : CURRENCY)}, \texttt{module Dollar =
  (Float : CURRENCY)}. Euro.t and Dollar.t incompatible.
\begin{lstlisting}
module type CLIENT =
sig 
...
end;; 

module type BANK =
sig
 include CLIENT
end;;
\end{lstlisting}
include will inherit all vals declared in CLIENT sig. Can overshadow
by redeclaring same name.
\\\textbf{Functor}, takes in module as input
\begin{lstlisting}
module Old_Bank (M : CURRENCY) : (BANK with type currency = M.t) = 
struct 
 type currency = M.t
 type t = {mutable balance : currency}
end;;
\end{lstlisting}
\color{RoyalPurple}
\section{Continuations}
Representation of execution state of a program (ex. call stack) at a
pt in time. Save state and restore later.
\\ Every function can be written tail-recursively
\\ To re-write a fun tail-recursively, add additional arg, a
continuation (like acc). Base case calls continuation, recursive
builds up cont.
\vspace{-1 em}
\paragraph{Tail-recursion} Continuation is a functional accumulator,
represents call stack built when recursively calling func and builds
final result
\vspace{-1 em}
\paragraph{Failure Continuation} Continuation tells you what to do
upon \textbf{failure}
\begin{lstlisting}
let rec find_tr p t cont = match t with 
 | Empty -> cont ()
 | Node(l, d, r) -> 
  if (p d) then Some d
  else find_tr p l (fun () -> find_tr p r cont)

let find' p t = find_tr p t (fun () -> None)
\end{lstlisting}
\vspace{-1 em}
\paragraph{Success Continuation} Continuation keeps track of what to
do on \textbf{success}, builds final result
\begin{lstlisting}
let rec findAll' p t sc = match t with 
 | Empty -> sc []
 | Node(l,d,r) -> 
  findAll' p l 
   (fun el -> findAll' p r 
    (fun er -> if (p d) then sc (el@(d::er)) else sc (el@er)))
\end{lstlisting}
\color{Green}
\section{Lazy Programming}
Eager: Eval by call-by-value, variables bound to vals
\\ \texttt{let x = horribleComp (345) in 5}, this evals horribleComp
(345), binds xx  to it and evals 5
\\ It is easier to reason about eager comp, clear when things eval but
may eval things \textbf{never needed}
\\ Lazy Computation: Suspend computation until needed
\\ Memoize results, demand-driven
\\ Good for \textbf{infinite data and interactive data}
\\ Ex lists are finite. Can pattern match lists. Reason about lists
via induction.
\\ We \textbf{don't} construct infinite data, we define
\textbf{observations} we make about them
\\ Given stream we can ask for head of stream and tail (rest of
stream)
\\ Can we always make an observation? Does it eventually terminate?
No, prog may run forever. Prog should remain \textbf{productive}
though, i.e. can make observation at each step
\\ We can suspend eval of an expr
\begin{lstlisting}
type 'a susp = Susp of unit -> 'a
let force (Susp f) = f ()
let x = Susp(fun () -> 1 + 2) in
force x
\end{lstlisting}
Infinite data streams
\begin{lstlisting}
type 'a str =
{hd:'a ; tl ('a str) susp}

let rec numsFrom n = 
{hd = n ; 
 tl = Susp (fun () -> numsFrom (n+1))}
\end{lstlisting}
Get head using observation \texttt{hd}, getting tail gives suspended
stream. Want more elements, ask for more
\color{Black}
\section{Language Design}
3 key qs: Syntactically legal exprs? \textbf{Grammar.} Well-typed
exprs? \textbf{Static semantics.} How is expr executed?
\textbf{Dynamic semantics.}
\vspace{-1 em}
\paragraph{Syntactically Legal exprs}
Defined inductively by: number $n$ is an expr, bools \texttt{true} and
\texttt{false} are exprs, $e_1$, $e_2$ exprs then so is $e_1 \ op \
e_2$ with op $=\{+,=,-,*,<\}$, if $e$, $e_1$ and $e_2$ are exprs then
\texttt{if e then e1 else e2} is an expr
\\ Alternatively, Backus-Naur Form (BNF)
\\ operations op ::== $+|-|*|<|=$
\\ expressions e ::== $n | e_1 \ op \ e_2 | true | false | $if $e$ then
$e_1$ else $e_2 | x |$ let $x = e_1$ in $e_2$ end  $|fn\ y $=> $e |
e_1 \ e_2 | fn \ y$ => $e | e_1 e_2 \Downarrow v$
\\ In ocaml:
\begin{lstlisting}
type primop = Equals | 
LessThan | Plus | Minus | Times
type exp =
| Int of int
| Bool of bool
| If of exp * exp * exp
| Primop of primop * exp list
| Let of dec * exp
and dec = Val of exp * name
\end{lstlisting}
\vspace{-1 em}
\paragraph{Values}
Values v ::== $n | true | false$
\\ $e \Downarrow v$ means expr e evals to val v
\textbf{Formalizing}
\begin{minipage}{.2\textwidth}
\begin{prooftree}
  \AxiomC{}
  \RightLabel{B-VAL}
  \UnaryInfC{$v \Downarrow v$}
\end{prooftree}
\end{minipage}
\begin{minipage}{.2\textwidth}
\begin{prooftree}
  \AxiomC{$e \Downarrow true$}
  \AxiomC{$e_1 \Downarrow v$}
  \RightLabel{B-IFTRUE}
  \BinaryInfC{if $e$ then $e_1$ else $e_2 \Downarrow v$}
\end{prooftree}
\end{minipage}
\begin{minipage}{.15\textwidth}
\begin{prooftree}
  \AxiomC{$e \Downarrow false$}
  \AxiomC{$e_2 \Downarrow v$}
  \RightLabel{B-IFFALSE}
  \BinaryInfC{if $e$ then $e_1$ else $e_2 \Downarrow v$}
\end{prooftree}
\end{minipage}
~
\begin{minipage}{.2\textwidth}
\begin{prooftree}
  \AxiomC{$e_1 \Downarrow v_1$}
  \AxiomC{$e_2 \Downarrow v_2$}
  \RightLabel{B-OP}
  \BinaryInfC{$e_1 \ op \ e_2 \Downarrow \overline{v_1 \ op \ v_2}$}
\end{prooftree}
\end{minipage}
\begin{minipage}{.2\textwidth}
\begin{prooftree}
  \AxiomC{$e_1 \Downarrow v_1$}
  \AxiomC{$[v_1/x]e_2 \Downarrow v$}
  \RightLabel{B-LET}
  \BinaryInfC{let $x = e_1 $ in $e_2$ end $\Downarrow v$}
\end{prooftree}
\end{minipage}
\begin{minipage}{.2\linewidth}
  \begin{prooftree}
    \AxiomC{}
    \RightLabel{B-FN}
    \UnaryInfC{$fn \ x$ => $e \Downarrow fn \ x$=> $e$}x
  \end{prooftree}
\end{minipage}
% \begin{minipage}{.05\linewidth}
%   \begin{prooftree}
%     \AxiomC{$e_1 \Downarrow fn \ x$ => $e$}
%     \AxiomC{$e_2 \Downarrow v_2$}
%     \AxiomC{$[v_2/x]e \Downarrow v$}
%     \RightLabel{B-APP}
%     \TrinaryInfC{$e_1e_2 \Downarrow v$}
%   \end{prooftree}
% \end{minipage}
\begin{lstlisting}
  let rec eval e = match e with
| Int _ -> e
| Bool _ -> e
| If (e , e1 , e2 ) ->
( match eval e with
| Bool true -> eval e1
| Bool false -> eval e2
| _ -> raise ( Stuck " guard is not a bool " ) )
\end{lstlisting}
Formal description advantages:
\\ Coverage: $\forall$ expressions $e \exists$ eval rule. Determinacy:
if $e \Downarrow v_1$ and $e \Downarrow v_2$ then $v_1=v_2$. Value
soundness, if $e \Downarrow v$ then $v$ is a value

\vspace{-1 em}
\paragraph{Well-typed expressions}
Static type checking. Types classify exprs based on what they compute.
\\ Types $T$ ::= $int | bool | T_1 \rightarrow T_2 | T_1 \times T_2 |\alpha$
\\ \textbf{Typing in context} Given assumptions $x_1:T_1, \ldots,
x_n:T_n$ -> $\Gamma$, expr $e$ has type $T$. i.e. $\Gamma \vdash e:T$
\begin{minipage}{.2\textwidth}
\begin{prooftree}
  \AxiomC{}
  \RightLabel{T-T}
  \UnaryInfC{$\Gamma \vdash$true : bool}
\end{prooftree}
\end{minipage}
\begin{minipage}{.2\textwidth}
\begin{prooftree}
  \AxiomC{}
  \RightLabel{T-NUM}
  \UnaryInfC{$\Gamma \vdash$ n : int}
\end{prooftree}
\end{minipage}
\begin{minipage}{.2\textwidth}
\begin{prooftree}
  \AxiomC{}
  \RightLabel{T-F}
  \UnaryInfC{$\Gamma \vdash$false : bool}
\end{prooftree}
\end{minipage}
\begin{minipage}{.2\textwidth}
\begin{prooftree}
  \AxiomC{$\Gamma \vdash e_1$:int}
  \AxiomC{$\Gamma \vdash e_2$:int}
  \RightLabel{T-PLUS}
  \BinaryInfC{$\Gamma \vdash e_1 + e_2$ : int}
\end{prooftree}
\end{minipage}
\begin{minipage}{.2\textwidth}
\begin{prooftree}
  \AxiomC{$\Gamma \vdash e_1$:T}
  \AxiomC{$\Gamma \vdash e_2$:T}
  \RightLabel{T-EQ}
  \BinaryInfC{$\Gamma \vdash e_1 = e_2$ : bool}
\end{prooftree}
\end{minipage}
\begin{minipage}{.2\textwidth}
\begin{prooftree}
  \AxiomC{$\Gamma \vdash e$:bool}
  \AxiomC{$\Gamma \vdash e_1$:T}
  \AxiomC{$\Gamma \vdash e_2$:T}
  \RightLabel{T-IF}
  \TrinaryInfC{$\Gamma \vdash$ if $e$ then $e_1$ else $e_2$ : T}
\end{prooftree}
\end{minipage}
\begin{minipage}{.2\linewidth}
  \begin{prooftree}
    \AxiomC{$\Gamma(x)=T$}
    \RightLabel{T-Var}
    \UnaryInfC{$\Gamma \vdash x : T$}
  \end{prooftree}
% \end{minipage}
% \begin{minipage}{.2\linewidth}
  \begin{prooftree}
    \AxiomC{$\Gamma \vdash e_1 : T_1$}
    \AxiomC{$\Gamma, x:T_1 \vdash e_2:T$}
    \RightLabel{T-LET}
    \BinaryInfC{$\Gamma \vdash$ let $x=e_1$ in $e_2$ end $:T$}
  \end{prooftree}
  \begin{prooftree}
    \AxiomC{$\Gamma,x:\alpha \vdash e:T_2 \Rightarrow T/C$}
    \RightLabel{T-FN}
    \UnaryInfC{$\Gamma \vdash fn \ x$=> $\alpha \rightarrow T/C$}
  \end{prooftree}
  % \begin{prooftree}
  %   \AxiomC{$\Gamma \vdash e_1 : \Rightarrow T_1/C_1$}
  %   \AxiomC{$\Gamma \vdash e_2 \Rightarrow T_2/C_2$}
  %   \RightLabel{T-APP}
  %   \BinaryInfC{$\Gamma \vdash e_1 e_2 \Rightarrow \alpha/C_1 \cup C_2
  %     \cup \{T_1 = (t_2 \rightarrow \alpha)\}$}
  % \end{prooftree}
\end{minipage}
\\
\textbf{Type checking} e:T Given exp $e$ and type $T$
, check that $e$ has type $T$. Are all substitution instances of $e$
well-typed? \textbf{Type Inference} e:T Given expr
$e$, infer its type $T$. Is some substitution instance of $e$ well-typed?
\vspace{-1 em}
\paragraph{Free Variables}
$FV(e)$ returns the set of free var names occurring in expr
$e$. Defined inductively based on structure of expr $e$.
\vspace{-1 em}
\paragraph{Substitution}[e'/x]e Replace all \textbf{free} occurrences
of x in expr $e$ by $e'$
\vspace{-1 em}
\paragraph{Type Inference} How to get type of expr $e$ given $\Gamma$?
Analyze $e$ following typing rules. Analyze $e$ recursively, if
lacking type info, introduce type var and possible constraints. $T$
may contain type vars. Solve constraints to see if $e$ is
well-typed. Solving gives a type substitution $\sigma$.
\begin{itemize}
  \setlength\itemsep{-0.5 em}
\item \texttt{fn x => x + 1} type $\alpha \rightarrow int/\{\alpha =
  int\}$
\item \texttt{fn f => fn x => f (f x)} type $\alpha \rightarrow \beta
  \rightarrow \alpha_1, \{\alpha = \beta \rightarrow a_0, \alpha =
  \alpha_0 \rightarrow \alpha_a\}$
\end{itemize}
\vspace{-1 em}
\paragraph{Constraint Solving via Unification} Two types unifiable if
exists instantiation $\sigma$ for type vars in both types such that
they are syntactically equal.
\\ Simplify set of contrainsty by rewriting constraints and getting
rid of useless ones
%%% Local Variables:
%%% mode: latex
%%% TeX-master: "Final"
%%% End:
