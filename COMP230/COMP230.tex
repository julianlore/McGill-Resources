% Created 2017-10-06 Fri 21:36
% Intended LaTeX compiler: pdflatex
\documentclass[11pt]{article}
\usepackage[utf8]{inputenc}
\usepackage[T1]{fontenc}
\usepackage{graphicx}
\usepackage{grffile}
\usepackage{longtable}
\usepackage{wrapfig}
\usepackage{rotating}
\usepackage[normalem]{ulem}
\usepackage{amsmath}
\usepackage{textcomp}
\usepackage{amssymb}
\usepackage{capt-of}
\usepackage{hyperref}
\usepackage{wasysym}
\date{\today}
\title{}
\hypersetup{
 pdfauthor={},
 pdftitle={},
 pdfkeywords={},
 pdfsubject={},
 pdfcreator={Emacs 25.3.1 (Org mode 9.1.1)}, 
 pdflang={English}}
\begin{document}

\tableofcontents

\section{Lecture 1 \textit{<2017-09-05 Tue>}}
\label{sec:org7f0b9ce}
Logic \& Computability
Prof. Dirk Schlimm
\begin{itemize}
\item Find out what Schlimm means for the next lecture
\item Great for people in computer science, but everyone else too
\begin{itemize}
\item Essential material that everyone should know
\item Stable material, as the material is old
\item Very abstract and technical material, even if it does not require a solid mathematical background
\item \uline{Hard} course
\begin{itemize}
\item Important to give feedback to the professor
\end{itemize}
\end{itemize}
\end{itemize}
This course complements the textbook, Godel, Escher Bach.

\subsection{Homework}
\label{sec:orgd59e5a6}
is not graded, just checked if done.
\begin{itemize}
\item Why?
\begin{itemize}
\item To motivate us to do homework exercises
\item Practice is important, the course is hard
\item TAs don't need to correct them, so they can hold more office hours
\end{itemize}
\end{itemize}

Discussion board on MyCourses. Do not email the professor, ask questions on discussion board so everyone can see the answer (incase they have the same question).

Some times there will be intentional mistakes on the board.
\begin{itemize}
\item To make it easier to ask questions
\item To motivate us to pay attention
\end{itemize}

There are no stupid questions, even if you ask the same question as the person before you. Perhaps the professor's answer was unclear.
The most stupid question is the one not being asked.
\subsection{Quizzes}
\label{sec:org1e55243}
\begin{itemize}
\item 3 quizzes throughout this course
\item Dates will be on the schedule on professor's website
\item In class, 15-20 minutes long
\item Each one is graded and worth 10\%
\item Fairly straightforward, some are even definitions
\item To make sure you've done your work
\end{itemize}

Midterm is 25\%, Final is 40\%

\subsection{Proofs}
\label{sec:org842678b}
We will see lots of proofs, different kinds of proofs.
\begin{itemize}
\item Direct: Go from assumption towards the theorem
\item Indirect: Negation of claim \(\rightarrow\) contradiction \(\rightarrow\) claim
\begin{itemize}
\item Sometimes called proof by contradiction
\end{itemize}
\item Biconditional: \(2\) claims, \(p,q\)
\begin{itemize}
\item Start with \(p\) and prove \(q\) but also start with \(q\) and prove \(p\)
\end{itemize}
\item By cases: Split claim into several cases
\begin{itemize}
\item case 1, case 2, case 3
\item Each one proves the same conclusion
\item If the cases were exhaustive, then you have proved the claim
\end{itemize}
\item Induction (to be taught next lecture)
\end{itemize}

So you can split a proof into subproofs of these kinds.

\subsubsection{Example Proofs}
\label{sec:org4a0f892}
\begin{enumerate}
\item Thm
\label{sec:orgf81e8a5}
\(\sqrt{2}\) is nor rational.
\begin{itemize}
\item Rational: Fractions: \(\frac{x}{y}\)
\end{itemize}
If you have a square with sides of length 1, the length of the diagonal is \(\sqrt{2}\)
Pythagoras proved this.
\begin{enumerate}
\item Def 1
\label{sec:orgb51a455}
A natural number \(a\) is \uline{even} if and only if (iff) \(\exists\) a natural number \(b\), such that \(a=2b\).
\item Lemma 1
\label{sec:org4d8e5fd}
For any number \(a\), \(a^2\) is even \uline{iff} \(a\) is even.
\begin{itemize}
\item Biconditional, since iff
\end{itemize}
\begin{enumerate}
\item Proof:
\label{sec:orgcd56f7f}
\(\leftarrow\) Assume: \uline{\(a\) is even}.
So there is: \(a=2b\) (by def. 1)

\(a^2=(2b)^=4b^2=2(\underbrace{2b^2}_c)\) (square)
So, \uline{\(a^2\) is even}, since it is \(2\) times \(c\), a natural number.

\(\rightarrow\) (DIY)
Assume: \uline{\(a^2\) is even}.
So there is: \(a^2=2b\) (by def. 1)
\end{enumerate}

\item Lemma 2
\label{sec:org3753f5b}
For any rational number \(x\), there are natural numbers \(a\) and \(b\), \uline{not both even}, s.t. \(x=\frac{a}{b}\)
\begin{itemize}
\item If they were both even, you could simplify the fraction by dividing by 2.
\end{itemize}
Proof omitted for this lemma.
\item Proof of Thm:
\label{sec:org8879f6a}
Indirect proof. (Contradiction)
Assume (for reductio/contradiction): \(\sqrt{2}\) is rational.
By Lemma 2, \(\exists\) natural numbers \uline{\(a\) and \(b\) not both even} s.t. \(\sqrt{2}=\frac{a}{b}\)

Square: \(2=\left(\frac{a}{b}\right)^2=\frac{a^2}{b^2}\)
\(a^2=2b^2\)

So \(a^2\) is even (by def. 1)
\uline{\(a\) is even} by (lemma 1)
If \(a\) is even, we can write: \(a=2c\) (by def 1)
Square: \(a^2=(2c)^2=4c^2=a^2\)

\(4c^2=2b^2\)

Divide by 2: \(2c^2=b^2\)
So \(b^2\) is even (def. 1)
\uline{\(b\) is even} (by lemma 1)

Contradiction! Assumption is false, therefore \(\sqrt{2}\) is \uline{not} rational. \(\Box\)
\end{enumerate}
\item {\bfseries\sffamily TODO} Download Handout and read it
\label{sec:org5a86884}
\end{enumerate}
\section{Lecture 2 \textit{<2017-09-07 Thu>}}
\label{sec:org5c109c6}
Last class we talked about proofs \& types of proofs.
Next week we'll be talking about sets and countability and comparing them all. Will talk about density of rationals and irrationals, to say which is bigger.
\subsection{Things that are infinite}
\label{sec:orge6732fb}
\begin{itemize}
\item Natural numbers
\item Rational numbers
\item Infinite lists
\end{itemize}
\begin{enumerate}
\item How do you prove things about infinitely large things?
\label{sec:org7a6ce04}
\begin{itemize}
\item Counter example
\begin{itemize}
\item All swans are white
\begin{itemize}
\item Show one that isn't white
\end{itemize}
\end{itemize}
\item Pick an arbitrary example and show that it works for that
\begin{itemize}
\item Use particular properties about an arbitrary object to show that something works for all of them
\end{itemize}
\end{itemize}
\end{enumerate}
\subsection{Mathematical induction}
\label{sec:orgbf7b8bd}
\begin{itemize}
\item Inference (step):
\begin{itemize}
\item Certain number of assumptions/premises \(A_n\ldots A_n\)
\item Conclusion
\end{itemize}
\end{itemize}
\subsubsection{Deduction:}
\label{sec:org71a0b22}
\begin{itemize}
\item It is impossible for the premises of an inference step to be \uline{true} and the conclusion \uline{false}.
\item The conclusion follows \uline{necessarily} from the premises. (reformulation of above)
\end{itemize}
\begin{enumerate}
\item e.g.
\label{sec:orgd66fa5e}
\(\frac{\text{if }A \text{ then }B \hspace{5 pt}A}{B}\)
\begin{itemize}
\item MODUS PONENS (type of deductive inference)
\end{itemize}
\end{enumerate}
\subsubsection{Induction:}
\label{sec:orgb235902}
\begin{itemize}
\item The premises make the conclusion \uline{more likely}
\item My cat is smart, my friend's cat is smart, my parent's cat is smart, so all cats are smart
\begin{itemize}
\item Inductive argument, makes it more likely, but doesn't see them all
\end{itemize}
\end{itemize}
\subsubsection{Inductive or Deductive?}
\label{sec:org69168ef}
\begin{enumerate}
\item Claim
\label{sec:orgcb93829}
Let \(n\) be the number of points on a circle. Then the number of regions obtained by pairwise connecting each point is \(R=2^{n-1}\)
\begin{enumerate}
\item Argument
\label{sec:orgba23f41}
\begin{center}
\begin{tabular}{ll}
\(n\) & \(R\)\\
\hline
\(1\) & \(1=2^0\)\\
\(2\) & \(2=2^1\)\\
\(3\) & \(4=2^2\)\\
\(4\) & \(8=2^3\)\\
\(5\) & \(16=2^4\)\\
\(6\) & \(31\)\\
\end{tabular}
\end{center}

\begin{center}
\includegraphics[width=.9\linewidth]{./Images/i1.png}
\end{center}   

\begin{itemize}
\item Inductive argument
\item What was wrong with the argument?
\begin{itemize}
\item We saw a pattern, but\ldots{}
\begin{itemize}
\item There's no reason for the jump from each \(n\) to have something in common
\item If they had something in common, then it would continue holding for the next one
\end{itemize}
\item Therefore induction makes the premise more likely
\begin{itemize}
\item But does not establish it deductively
\end{itemize}
\item So in order to rigorously prove something inductively, we need mathematical induction
\end{itemize}
\end{itemize}
\end{enumerate}
\end{enumerate}

\subsection{Handout}
\label{sec:orga42c044}
\subsubsection{Recursive (inductive) definition:}
\label{sec:orge6c63c0}
\begin{enumerate}
\item Base clause(s) defines basic elements.
\item Inductive clause(s): How to build up complex elements from parts
\item Final clause: Nothing else is an element (bookkeeping)
\end{enumerate}
\begin{enumerate}
\item E.g.
\label{sec:orga3f4927}
\begin{enumerate}
\item \(\mathbb{N}\)
\begin{itemize}
\item Base clause \(0\) is in \(\mathbb{N}\)
\item Inductive clause: if \(x \in \mathbb{N}\) then \(s(x)\) (successor of \(x\)) then \(s(x)\) is in \(\mathbb{N}\)
\item Final clause: Nothing else is in \(\mathbb{N}\)
\item So natural numbers are:
\begin{itemize}
\item \(0, s(0), s(s(0)), \ldots\)
\end{itemize}
\end{itemize}
\item Even numbers or odd numbers
\begin{itemize}
\item Take successor of successor, take \(0\) as base clause for even, \(1\) for odd
\end{itemize}
\item Lists
\begin{itemize}
\item Empty list is a list
\item What you get from adding to a list is also a list
\end{itemize}
\item Dominoes
\begin{itemize}
\item When you have a domino, you can place one 2 cm behind it
\item Push first one, they all fall
\begin{itemize}
\item To prove they all fall, have to show they all have a certain amount of space between them
\begin{itemize}
\item Relates to proof by mathematical induction
\end{itemize}
\end{itemize}
\end{itemize}
\end{enumerate}
\item Proof by mathematical induction
\label{sec:org9d7ec22}
\begin{enumerate}
\item Base case: Show that the property holds of the basic elements.
\item Inductive step:
\begin{enumerate}
\item Assume that the property holds for some element \(n\) (Inductive Hypothesis)
\item \uline{Show}: holds for elements generated from \(n\) by inductive clauses.
\end{enumerate}
\item Conclusion: Property holds \uline{for all} elements.
\end{enumerate}
This is \textbf{deductive inference}!
What are the premises?
\begin{itemize}
\item For natural numbers:
\begin{itemize}
\item \(\frac{\overbrace{P(0)}^{\text{Base case}} \overbrace{P(n)}^{\text{IH}}\overbrace{\to}^{\text{Ind. step}} P(s(n))}{\forall x P(x)}\)
\end{itemize}
\end{itemize}
\item Variant (strong/complete induction):
\label{sec:org2738f4b}
\begin{itemize}
\item \uline{Ind. Step.}
\begin{enumerate}
\item Assume that \(P\) holds for all elements \uline{less than} \(n\)
\item Show: \(P\) holds of \(n\)
\end{enumerate}
\item \uline{No base case}
\item See example 5.5!
\end{itemize}
\end{enumerate}
\subsubsection{Theorem}
\label{sec:org67afc41}
For any nat. number \(n\geq 1\), the sum \(\underbrace{1+2+\ldots+n}_{\sum_i=1^n i}=\frac{n(n+1)}{2}\)
(If you do a proof for your homework or on an exam, always include many details. You can even use a template to structure your proofs the same way, useful for steps for induction.)
\begin{enumerate}
\item Proof (by math. ind)
\label{sec:org5ad39f7}
\begin{enumerate}
\item Base case: Show claim holds for \(n=1=\frac{1(1+1)}{2}\)
\item Ind. step.
\begin{enumerate}
\item I.H. The claim holds for \(m\): \(\sum_{i=1}^m i = \frac{m(m+1)}{2}\)
\item Show: The claim holds for \(m+1\)
\end{enumerate}
\end{enumerate}
2 strategies, either \(\frac{n(n+1)}{2}\to {1+2+\ldots+n}\) or \({1+2+\ldots+n}\to \frac{n(n+1)}{2}\). Will be doing 2nd.
\(1+2+\ldots+(m+1)=\sum_{i=1}^{m+1}i=\sum_{i=1}^m i + (m+1)\)

\(=\frac{m(m+1)}{2}+(m+1)\) (by I.H.)

\(=\frac{m(m+1)+2m+2}{2}=\frac{(m+1)(m+2)}{2}\)
\begin{enumerate}
\item Conclusion: The claim holds \uline{for all} \(n\geq 1\) \(\Box\)
\end{enumerate}
\end{enumerate}
\section{Lecture 3 \textit{<2017-09-12 Tue>}}
\label{sec:orge932287}
\subsection{Set theory}
\label{sec:org9c4efbd}
All that is being said here is taken from the reading mathematical introduction to logic chapter zero.

A \uline{set} is a thing with elements. We can present sets in two ways:
\begin{itemize}
\item Extensional:
\begin{itemize}
\item Presentation
\item \(\{1,2,3\}\)
\end{itemize}
\item Intensional:
\begin{itemize}
\item Given a set \(A\), and a property \(P\): \(\{x\in A|P(x)\}\)
\end{itemize}
\end{itemize}
\subsubsection{Ex}
\label{sec:org647cd8c}
\(\mathbb{N}\): the set of natural numbers.

\(D=\{x\in \mathbb{N}|x \text{ is prime}\}=\{2,3,5,7,11, \ldots\}\)
\subsubsection{Definitions}
\label{sec:org96e068d}
\begin{itemize}
\item \(A\subseteq B \iff \forall x, x\in A \implies x \in B\)
\item \(A=B \iff (A\subseteq B) \wedge (B\subseteq A)\)
\item \(A \underbrace{\subset}_{\text{Proper subset}} B \iff (A\subseteq B)\wedge (A\neq B)\)
\item Empty set: \(\emptyset, \{\}\)
\begin{itemize}
\item When is \(x\in \emptyset\)? Never.
\item \(\emptyset \subseteq X\)? Always.
\begin{itemize}
\item Since all elements of the empty set are in \(X\).
\end{itemize}
\item \(\emptyset \in X\)?. If \(X\) contains \(\emptyset\).
\begin{itemize}
\item E.g. \(X=\{\{\},4\}\)
\end{itemize}
\end{itemize}
\end{itemize}
\(A=\{2,4,8\}, B=\{a,4,z\}\)
\begin{itemize}
\item \(\underbrace{A\cap B}_{\text{intersection}}: \forall x, (x\in A) \wedge (x\in B)\)
\begin{itemize}
\item \(A\cap B = \{4\}\)
\end{itemize}
\item \(\underbrace{A\cup B}_{\text{union}}: \forall x, (x\in A) \vee (x\in B)\)
\begin{itemize}
\item \(A\cup B = \{2,4,8,a,z\}\)
\end{itemize}
\item complement: \(\bar{A}:\) all elements that are not in \(A\) (from the \uline{universe of discourse}, the universe we're talking about)
\item Power set \(\frak{P}(A):\) the set of all subsets of \(A\)
\begin{itemize}
\item E.g. \(\frak{P}(B)=\{\emptyset,\{a\},\{4\},\{z\},\{a,4\},\{a,z\},\{4,z\},\{a,4,z\}\}\)
\item If \(A\) has \(n\) elements, \(\frak{P}(A)\) has \(2^n\) elements.
\end{itemize}
\item Is \(\{\emptyset,a\}\subseteq\{a,4,z\}\)? No.
\end{itemize}
\subsection{Tuples:}
\label{sec:orga2bce36}
Like sets, but order matters.
\begin{itemize}
\item Ordered pair: \(\langle a,b \rangle = \{\{a\},\{a,b\}\}\)
\item \(\langle a,4,z \rangle \neq \langle 4,a,z \rangle\)
\end{itemize}
\subsubsection{Cross-product}
\label{sec:org3968b3a}
\(A \times B \iff \{\langle x,y \rangle|x \in A \wedge y\in B\}\)
\begin{itemize}
\item If \(A\) has \(n\) elements, \(B\) has \(m\) elements
\item then \(A \times B\) has \(n\cdot m\) elements
\item and there are \(2^{n \cdot m}\) relations between \(A\) and \(B\)
\begin{itemize}
\item Since this is essential just the cardinality of the power set of the cross product
\item E.g. \(n=5, m=5\). 2 pairs of 5 friends. How many relations are possible? \(2^{25}=33,554,432\)
\end{itemize}
\end{itemize}
\subsubsection{Relations}
\label{sec:orgb0f0608}
\begin{itemize}
\item \(A=\{\text{John, Paul, George}\}\)
\item \(B=\{\text{guitar, bass}\}\)
\item \(\{\langle \text{John, guitar}\rangle, \langle \text{Paul, bass} \rangle, \langle \text{George, guitar} \rangle\}=R_1\)
\item \(R_2 = \{\langle \text{John, bass} \rangle\}\)
\end{itemize}
A \uline{relation} \(R\) on \(A\) and \(B\) is a subset of \(A\times B\).
\begin{itemize}
\item Elements of relations are tuples.
\end{itemize}
\uline{Domain} of a relation \(R: \{a| \text{there is a b, s.t. }\langle a,b \rangle \in R\}\)
\begin{itemize}
\item domain of \(R_1: \{\text{John, Paul, George}\}\)
\item domain of \(R_2: \{\text{John}\}\)
\end{itemize}
\uline{Range} of a relation \(R\): \(\{b| \text{there is an a, s.t. } \langle a,b
 \rangle \in R\}\)
\begin{enumerate}
\item Functions
\label{sec:org56c9dc1}
A total \uline{function} \(f:A\to B\) is a binary relation \(R\), on \(A\) and \(B\) such that.
\begin{itemize}
\item It is \uline{single-valued}
\begin{itemize}
\item Every element in \(A\) is mapped to exactly one element in \(B\)
\end{itemize}
\item The domain of \(R\) is \(A\)
\end{itemize}
\begin{center}
\includegraphics[width=.9\linewidth]{./Images/i2.png}
\end{center}
\begin{enumerate}
\item Definitions
\label{sec:org8d307b4}
\begin{itemize}
\item A function is \uline{injective} (one-to-one), if each element in the range is mapped to by exactly one element.
\begin{itemize}
\item To show this: Assume \(f(x)=f(y)\)
\begin{itemize}
\item Show \(x=y\)
\end{itemize}
\item So you don't have the situation that: \begin{center}
\includegraphics[width=.9\linewidth]{./Images/i3.png}
\end{center}
\end{itemize}
\end{itemize}

\begin{center}
\includegraphics[width=.9\linewidth]{./Images/i4.png}
\end{center} 

\begin{itemize}
\item A function is \uline{surjective} if the \(range=codomain\).
\item A function that is both injective and surjective is \uline{bijective}.
\end{itemize}
\end{enumerate}
\end{enumerate}
\section{Lecture 4 \textit{<2017-09-14 Thu>}}
\label{sec:org9775f6b}
\subsection{Recap}
\label{sec:org6931028}
We talked about sets last class, such as:
\begin{itemize}
\item \(\{1,4,z\}\)
\item \(|\{1,4,z\}| = 3\) (Cardinality)
\end{itemize}

\noindent\rule{\textwidth}{0.5pt}
Are there more students or chairs in this class? 
\begin{itemize}
\item There are more chairs.
\item Matched students with chairs and to see what is left
\item f(students) \(\to\) chairs
\begin{itemize}
\item Injective function (can't have 2 students on one chair)
\item Every element of the range must be mapped to something
\item No element in the range can map to two elements
\end{itemize}
\item \(\implies |S| \leq |C| \iff\) there is an injective function from \(S\) to \(C\).
\end{itemize}

Cantor: \(|A|=|B| \iff |A| \leq |B|\) and \(|B|\leq |A| \iff\) there is a bijection between \(A\) and \(B\).
\subsection{More on sets}
\label{sec:orga89b02f}
\subsubsection{Cardinalities}
\label{sec:org92d922d}
A set D is \uline{finite} if its cardinality is a natural number.
\begin{itemize}
\item \(D \leftrightarrow \{1,\ldots,n\}\) (bijective function with natural numbers exists)
\end{itemize}
A set is \uline{countably infinite} (denumerable), if it is equinumerous to \(\mathbb{N}\) (bijection from this set to all the natural numbers).
\begin{itemize}
\item \(E=\{2,4,6,8,\ldots\}\)
\begin{itemize}
\item \(|E| = |\mathbb{N}| = |\mathbb{Z}|=|\mathbb{Q}|<|\mathbb{R}|\)
\end{itemize}
\end{itemize}
\begin{center}
\begin{tabular}{llllll}
\(\mathbb{N}\) & \(1\) & \(2\) & \(3\) & \(4\) & \ldots{} \(n\)\\
\hline
\(E\) & \(2\) & \(4\) & \(6\) & \(8\) & \ldots{}\(2n\)\\
\end{tabular}
\end{center}
\(f(x)=2x, \mathbb{N}\to E\)

\(\mathbb{Z}=\{\ldots -3, -2, -1, 0, 1, 2, 3, \ldots\}\)
\begin{itemize}
\item Is this bigger than the cardinality of the natural numbers?
\begin{itemize}
\item No, it's the same size, bijection. Even to positive, odds to negative.
\end{itemize}
\end{itemize}

\begin{center}
\begin{tabular}{lllllll}
\(-3\) & \(-2\) & \(-1\) & \(0\) & \(1\) & \(2\) & \(3\)\\
\hline
\(5\) & \(3\) & \(1\) & \(0\) & \(2\) & \(4\) & \(6\)\\
\end{tabular}
\end{center}

\(\mathbb{Q}^+ = \{\frac{x}{y}|x,y \in \mathbb{N}\}\)
\begin{itemize}
\item Are there more?
\end{itemize}
\begin{center}
\begin{tabular}{rlllllll}
 & 1 & 2 & 3 & 4 & 5 & 6 & \ldots{}\\
\hline
1 & 1/1 & 1/2 & 1/3 & 1/4 & 1/5 & 1/6 & \ldots{}\\
2 & 2/1 & 2/2 & 2/3 & 2/4 & 2/5 & 2/6 & \ldots{}\\
3 & 3/1 & 3/2 & 3/3 & 3/4 & 3/5 &  & \\
4 & 4/1 & 4/2 & 4/3 & 4/4 &  &  & \\
5 & 5/1 & 5/2 & 5/3 &  &  &  & \\
6 &  &  &  &  &  &  & \\
\end{tabular}
\end{center}
There are duplicates here though. So, instead of counting left to right, count diagonally.
\begin{itemize}
\item i.e. 1:1/1, 2:1/2, 3:2/1, 4:3/1, 5:2/2, 6:1/3, \ldots{}
\end{itemize}

\(\mathbb{R} = \mathbb{Q} \cup \{\text{irrationals}\}\)
\begin{itemize}
\item What are real numbers? All numbers that can be expressed via decimal expansion.
\item \(x.xxxxx\ldots\)
\item Is this countably infinite? No.
\end{itemize}
\subsubsection{Proof (by contradiction):}
\label{sec:org5734de3}
Assume \(|\mathbb{N}|=|\mathbb{R}^{0.1}|\). 
(\(\mathbb{R}^{0.1} = \{x \in \mathbb{R}| 0<x<1\}\).)

Therefore, there is a bijection \(f: \mathbb{N} \to \mathbb{R}^{0.1}\)

Then, we can build the following table:
\begin{center}
\begin{tabular}{ll}
\(\mathbb{N}\) & \\
\hline
\(0\) & \(f(0)=0.12345\ldots\)\\
\(1\) & \(f(1)=0.33333\ldots\)\\
\(2\) & \(f(2)=0.5000 \ldots\)\\
\(3\) & \(f(3)=0.011101\ldots\)\\
\(4\) & \(\ldots\)\\
\(\ldots\) & \(\ldots\)\\
\(n\) & \(f(n)=0.112\ldots=z\)\\
\end{tabular}
\end{center}

Can you explain this table? Not really. Why?
\begin{itemize}
\item Construct new number \(z\):
\begin{itemize}
\item \(z=0.z_1z_2z_3z_4\ldots\)
\item Rule for constructing \(z\):
\end{itemize}
\end{itemize}
\begin{equation*}
z_i= \begin{cases}
1 & \text{if } f(i)_j \neq 1 \text{ where $f(i)_j$ is the ith digit in the decimal expansion of f(i)}
\\ 2 & \text{otherwise}
\end{cases}
\end{equation*}
\begin{itemize}
\item \(f(1)_1 = 3\)
\item \(f(2)_2 = 0\)
\item \(f(3)_3 = 1\)
\item \(\implies z=0.112\ldots\)
\item By construction, \(z\) is a real number between \(0\) and \(1\).
\item So it must be in the table, say in line \(n\).
\end{itemize}
What is \(z_n\)?
\begin{itemize}
\item Two cases:
\begin{itemize}
\item \(z_n = 2 = f(n)_n\) if \(f(n)_n = 1 \lightning\)
\item \(z_n = 1 = f(n)_n\) of \(f(n)_n \neq 1 \lightning\)
\item \(\implies\) contradiction!
\begin{itemize}
\item The assumption is false.
\end{itemize}
\end{itemize}
\end{itemize}
\section{Lecture 5 \textit{<2017-09-19 Tue>}}
\label{sec:orgbca7d8c}
\begin{itemize}
\item Quiz in one week
\item 20 minutes, in class
\item 8-10 questions, very simple
\begin{itemize}
\item Everything that was said in class
\item Everything done in the homework
\item Readings
\end{itemize}
\end{itemize}
\subsection{Recap}
\label{sec:org527a27e}
Last class, we proved:
(Size of Natural numbers) \(\aleph_0 < |\mathbb{R}|\)
\(\implies\) Diagonalization
\begin{itemize}
\item We had a table and changed every element on the diagonal in order to get a new element
\item We will see many more proofs by diagonalization
\item Homework question: \(|\frak{P}(\mathbb{N})|>|\mathbb{N}|\)
\begin{itemize}
\item In general though, \(|\frak{P}(x)|>|x|\) (Cantor's theorem)
\begin{itemize}
\item What does this imply? There are infinite amount of infinite cardinalities (power set is bigger, power set of the power set is even bigger, \ldots{})
\item \(|\mathbb{N}|<|\frak{P}(\mathbb{N})|=|\mathbb{R}|=2^{\aleph_0}<|\frak{P}(\frak{P}(\mathbb{N}))|\)
\end{itemize}
\end{itemize}
\end{itemize}

\subsection{Cardinality}
\label{sec:orgcb1ca7c}
Things that have the same cardinality:
\subsubsection{Countable:}
\label{sec:org0a64132}
\(|\mathbb{N}|\):
\begin{itemize}
\item E
\item \(\mathbb{Z}\)
\item \(\mathbb{Q}\)
\item English words
\item Sentences
\begin{itemize}
\item Finite objects that you can list
\item Why doesn't diagonalization work on sentences?
\end{itemize}
\item Programs
\item MIU strings
\item MIU theorems
\item If you can list them, they're countable
\end{itemize}
\subsubsection{Uncountable}
\label{sec:orgf333f04}
\(|\frak{P}(\mathbb{N})|\)
\begin{itemize}
\item \(\mathbb{C}\)
\item \(\mathbb{R}\)
\item Functions from \(\mathbb{N} \to \mathbb{N}\)
\begin{itemize}
\item From \(\mathbb{N} \to \{0,1\}\)
\end{itemize}
\end{itemize}

\noindent\rule{\textwidth}{0.5pt}
\subsection{Formal Systems (GEB CH. 1)}
\label{sec:org77f4e29}
\subsubsection{Examples}
\label{sec:org95f0d2b}
\begin{itemize}
\item Programming languages
\item Logic
\item Computation: TM
\item Formal arithmetic
\end{itemize}

\noindent\rule{\textwidth}{0.5pt}
2 things in formal systems:
\begin{itemize}
\item The distinction between the two is very important
\begin{itemize}
\item Important concepts in this course:
\begin{itemize}
\item Induction
\item Diagonalization
\item Distinction between these 2 things
\end{itemize}
\end{itemize}
\end{itemize}

\begin{center}
\begin{tabular}{ll}
Syntax & Semantics\\
\hline
- Grammar & - Meaning\\
- Formal Structure & - Context\\
\end{tabular}
\end{center}

13 -> What is this?
\begin{itemize}
\item 13 is a numeral
\begin{itemize}
\item The meaning of this numeral is the number 13 (abstraction)
\end{itemize}
\item Why this example? We looked at the syntax of 13 but we said it was the number 13 (the meaning)
\begin{itemize}
\item During everyday life we don't often make the distinction
\end{itemize}
\item dog
\begin{itemize}
\item Syntactically, has 3 letters
\item Semantically, has fur
\end{itemize}
\end{itemize}

\subsubsection{MIU-System:}
\label{sec:orgbdf10cf}
Alphabet: MIU

Strings (sequences of elements from the alphabet).

Rec. def: 
\begin{itemize}
\item Base clause: \(\emptyset\) is a MIU-string
\item Inductive clause:
\begin{enumerate}
\item If \(x\) is a MIU-String, then \(xM\) is a MIU string
\begin{itemize}
\item Is \(x\) an MIU-String? No, that's it's meaning, not it's syntax. It's a letter (also a meta-variablee)
\end{itemize}
\item \(xI\)
\item \(xU\)
\end{enumerate}
\item Final clause: Nothing else.
\end{itemize}
\begin{enumerate}
\item MIU-Theorems
\label{sec:org4b661f2}
\begin{enumerate}
\item Axiom: MI.
\item Inference rules:

I. xI \(\to\) xIU

II. Mx \(\to\) Mxx

III. xIIIy \(\to\) xUy

IV. xUUy \(\to\) xy

(for x,y MIU strings, possibly empty)
\end{enumerate}
\begin{enumerate}
\item Def. Derivation
\label{sec:org7e7216d}
A \uline{derivation} is a sequence of strings such that each element is either an axiom or obtained by applying an inference rule to an element earlier in the sequence

The last element in a derivation is a theorem.

\begin{itemize}
\item This is a recursive definition.
\begin{itemize}
\item Includes base clause
\item Inductive clause
\item Recursive clause
\end{itemize}
\end{itemize}
\item Ex.
\label{sec:org916d0db}
\begin{enumerate}
\item MI is a derivation
\item MIU by I on line 1.
\item MII by II on line 1.
\item MIUIU II on line 2.
\end{enumerate}
These are all theorems since they're the last element of a derivation.
\item Random theorems
\label{sec:orgdaeb4af}
\begin{itemize}
\item MIIII
\item MIIIIU
\item MIUU
\item MIUUIUU
\item MIIUU
\item They all have something in common, all start with M
\end{itemize}
\end{enumerate}
\item Reasoning
\label{sec:org730878a}
Reason inside (M-mode)
\begin{itemize}
\item Generate theorems "within" the formal system
\item Can be done by a machine
\item Object language
\end{itemize}
Outside a system (I-mode)
\begin{itemize}
\item Show properties \uline{of} the system, reason about it
\item Meta-language
\begin{itemize}
\item All theorems start with M
\item Looking on the outside
\item When do we use other languages to describe another language
\item Using English to talk about programming
\item Using English to talk about Mandarin
\item If you speak English, then you're reasoning inside
\end{itemize}
\end{itemize}
\item Bijection with Natural Numbers
\label{sec:org9c95d6e}
MIU strings are countably infinite. You can construct a bijection like:
\begin{enumerate}
\item M
\item I
\item U
\item MM
\item MI
\item MU
\item II
\item IM
\item IU
\item UU
\item \ldots{}
\end{enumerate}

\noindent\rule{\textwidth}{0.5pt}
MIU theorems are also countably infinite? Why?
\begin{itemize}
\item Subset of MIU strings
\item Why not finite? Inference II, can keep expanding
\end{itemize}
\item Theorem
\label{sec:orga128bb9}
All MIU-theorems begin with M.
\begin{itemize}
\item This is a proof about the MIU-system, not within
\end{itemize}
\begin{enumerate}
\item Proof
\label{sec:org5e922f8}
By induction (strong induction) on the length of derivations: (number of steps to derive)
\begin{itemize}
\item Base case: Derivation of length 1: MI (good)
\item Induction step:
\begin{itemize}
\item IH: The claim holds for all derivations of length \(<n\)
\item Show: The claim holds for a derivation of length \(n\)
\item Line n is either an axiom or derived by rule I, II, III or IV.
\end{itemize}
\item Case 1: Line n is an axiom: MI (you can write an axiom at any step, reverting back to MI)
\item Case 2: Line n is derived from an earlier line (say \(m<n\) ) by Rule I. By IH, the theorem in line \(m\) begins with M. Rule I doesn't change the first letter, so it is also an \(M\).
\item Case 3:
\item Case 4:
\item Case 5:
\item (DIY)
\end{itemize}
\end{enumerate}
\end{enumerate}
\section{Lecture 6 \textit{<2017-09-21 Thu>}}
\label{sec:orge779784}
\subsection{Quiz prep}
\label{sec:orgce48b6f}
\begin{itemize}
\item What is derivation?
\item What is a theorem?
\item How many infinite cardinalities are there?
\item Can a set have the same cardinality as its power set?
\item Is the empty set a subset of every set?
\item How do you prove something is inductive?
\end{itemize}
\subsection{Review}
\label{sec:orgb78c5c0}
\begin{itemize}
\item Is U a MIU-theorem?
\begin{itemize}
\item No, it doesn't start with M
\end{itemize}
\item Is MU a MIU-theorem?
\end{itemize}
\subsection{Decision Procedure}
\label{sec:org3ecf31b}
Guarantees a \uline{yes} or a \uline{no} answer in a \uline{finite} amount of time
\begin{itemize}
\item A set/question that has a decision procedure is \uline{decidable}
\item If given 2 functions with inputs and outputs, can we tell if they're identical? Is it decidable?
\begin{itemize}
\item No, infinite amount of inputs
\end{itemize}
\item Decision procedure for getting someone's cellphone number?
\begin{itemize}
\item Try all combinations until the phone rings, if no ring, no number
\item Not feasible, but we care what you can do in principal, as long as its finite
\end{itemize}
\item Given a computer program
\begin{itemize}
\item Can we decide if it terminates in 10 minutes?
\begin{itemize}
\item Yes, just wait
\end{itemize}
\item Can we decide if it terminates in finite time?
\begin{itemize}
\item No, if it doesn't stop, you'll never know
\end{itemize}
\end{itemize}
\end{itemize}
\subsection{pq-system}
\label{sec:org9f5defe}
\begin{itemize}
\item Alphabet: p, q, -
\item Axiom(s): xp-qx-
\begin{itemize}
\item What is x here? An arbitrary number of hyphens, meta-variable (used to describe system, not part of the system)
\item How many axioms? \(\aleph_0\), x can be uncountably many
\begin{itemize}
\item The written axiom is more like an axiom "template"
\end{itemize}
\item Is there a \textbf{decision procedure} to check if something is an axiom?
\begin{itemize}
\item Yes, just count number of hyphens.
\end{itemize}
\item Axioms in a formal system \textbf{have to be decidable}
\end{itemize}
\item IR: If xpyqz is a thm, then xpy-qz- is a thm
\begin{itemize}
\item E.g. \(\mbox{-}\mbox{-}p\mbox{-}\mbox{-}q\mbox{-}\mbox{-}\mbox{-}\mbox{-}\mbox{-}\)
\end{itemize}
\end{itemize}
\subsubsection{Interpretation:}
\label{sec:orgb43ba7a}
pq-system: p (plus) q (equals) \(\mbox{-}\) (1) \(\mbox{-}\mbox{-}\) (2) \(\mbox{-}\mbox{-} \mbox{-}\)(3)
\begin{center}
\begin{tabular}{llrrrll}
plus & equals & 1 & 2 & 3 & (Semantics) & Math structure \(\langle \mathbb{N},+, =\rangle\)\\
\hline
p & q & -\mbox{-} & \(\mbox{-}\mbox{-}\) & \(\mbox{-}\mbox{-}\mbox{-}\) & (syntax) & Typographical structure \(\langle \{ \mbox{-}, \mbox{-}\mbox{-}, \mbox{-}\mbox{-}\mbox{-} \}, p, q \rangle\)\\
\end{tabular}
\end{center}
\begin{itemize}
\item GEB: Calls this an Isomorphism
\begin{itemize}
\item Misleading, because in mathematics, it's a structure preserving bijection
\item \(\langle \mathbb{N},+ \rangle\) is isom \(\langle Even, + \rangle\) by \(f(x)=2x\)
\item \(a+b = c \iff f(a)+ f(b)=f(c)\)
\end{itemize}
\end{itemize}

\noindent\rule{\textwidth}{0.5pt}
Are \(\langle \mathbb{N},+ \rangle\) and \(\langle \mathbb{N},x\rangle\) isom?
\begin{itemize}
\item \(a+b = c \iff f(a) \times f(b) = f(c)\)
\item \(f(x)=2^x\)
\begin{itemize}
\item \(3+5=8 \to 2^3 \times 2^5 = 2^8\)
\item Does this work? No. Not surjective.
\item Is there a bijection?
\end{itemize}
\end{itemize}

If an interpretation makes all axioms and thm true, it is called a \uline{model}.
\begin{itemize}
\item Is our interpretation a model?
\item Yes, argue by saying it makes axioms true and IR keeps it true.
\begin{itemize}
\item E.g. \(\mbox{-}\mbox{-}p\mbox{-}\mbox{-}q\mbox{-}\mbox{-}\mbox{-}\mbox{-}\mbox{-}\)
\item \(2+3=5\)
\end{itemize}
\end{itemize}
\subsubsection{More Interpretations}
\label{sec:org28bd883}
Change p to times. It's still an interpretation, but not a model. It's false, as all axioms and theorems must be true.
\begin{itemize}
\item Keep p to plus, but change all dashes to negative integers. Is it also a model?
\begin{itemize}
\item Yes. We can still have multiple models for
\end{itemize}
\end{itemize}

pq-system: p (equals) q (taken from) \(\mbox{-}\) (2) \(\mbox{-}\mbox{-}\) (4) \(\mbox{-}\mbox{-} \mbox{-}\)(6)
\begin{itemize}
\item E.g. \(\mbox{-}\mbox{-}p\mbox{-}\mbox{-}q\mbox{-}\mbox{-}\mbox{-}\mbox{-}\mbox{-}\)
\item 6 = 4 taken from 10
\item Still a model!
\end{itemize}
\begin{itemize}
\item Formal system can have many models, just depends on interpretation
\end{itemize}
\subsection{geq-system:}
\label{sec:orgcfed6a7}
\begin{itemize}
\item Thms:
\begin{itemize}
\item \(\mbox{-}\) geq \(\mbox{-}\)
\item \(\mbox{-}\mbox{-}\) geq \(\mbox{-}\)
\item \(\mbox{-}\mbox{-}\mbox{-}\) geq \(\mbox{-}\mbox{-}\mbox{-}\)
\end{itemize}
\item Model: geq -> \(\geq\)
\begin{itemize}
\item \(\mbox{-}\) 1
\item \(\mbox{-}\mbox{-}\) 2
\item \(\ldots\)
\end{itemize}
\item Soundness: Every \textbf{thm} in a formal system is \textbf{true} under an interpretation
\item Completeness: Every \textbf{truth} in an interpretation is a \textbf{theorem}
\item Soundness and completeness relate semantic and syntactic notions with each other
\item Our interpretation is sound and complete
\item Different interpretation, if we make geq -> \(=\)
\begin{itemize}
\item Is it sound? No. Some theorems are true, but some, like \(\mbox{-}\mbox{-}\) geq \(\mbox{-}\) are not true
\item Is it complete? Yes. Every equality that can be expressed via this interpretation is a theorem.
\item Soundness and completeness are independent, so it was possible for us to get completeness and not soundness, but it's also possible to get the opposite \uline{(think of an example)}
\end{itemize}
\item model \(\iff\) sound
\item Syntax and semantics can be switched, which we'll see later
\begin{itemize}
\item We'll be looking at formal systems of numbers/arithmetic that mean different things
\end{itemize}
\end{itemize}
\subsection{Infinite Prime Numbers}
\label{sec:org89778a4}
Theorem: There are infinitely many prime numbers.

\subsubsection{Proof (by contradiction):}
\label{sec:orgd667624}
Assumption: There are \uline{finitely} many prime numbers:
\begin{itemize}
\item \(\{p_1, p_2, \ldots , p_n\}\)
\item So there is a greatest prime number, say \(p_n\)
\item Define: \(g=(p_1 \times p_2 \times \ldots \times p_n)+1\)
\begin{itemize}
\item Is \(g\) a prime number?
\begin{itemize}
\item Case 1: Yes. Then \(g>p_n\) \lightning
\item Case 2: \(g\) is not prime.
\begin{itemize}
\item But then it must be divisible by some prime number.
\item But, it \uline{cannot} be divisible by \(p_1,p_2,\ldots,p_n\), as there will be a remainder of \(1\) \lightning
\end{itemize}
\end{itemize}
\end{itemize}
\end{itemize}
\begin{itemize}
\item So, assumption is false.
\end{itemize}
\section{Lecture 7 \textit{<2017-09-26 Tue>}}
\label{sec:org1039239}
\subsection{Quiz review}
\label{sec:org355e918}
\begin{itemize}
\item Try to use diagonalization on rational numbers?
\begin{itemize}
\item Produced number may not be rational.
\end{itemize}
\item Show 2 sets have the same cardinality
\begin{itemize}
\item Prove there's a bijection
\end{itemize}
\item Show a set has less than or the same amount of cardinality
\begin{itemize}
\item Injection
\end{itemize}
\item Adding something to a set of size \(\aleph_0\)
\begin{itemize}
\item Size is still \(\aleph_0\)
\end{itemize}
\item What is a derivation?
\begin{itemize}
\item Sequence of formulas such that each element is an axiom or obtained from an axiom
\end{itemize}
\item Theorem?
\begin{itemize}
\item Last element in a derivation
\end{itemize}
\item When is a formal system complete?
\begin{itemize}
\item Every truth in the interpretation is a theorem
\end{itemize}
\item When is a formal system sound?
\begin{itemize}
\item Every theorem is a truth in the interpretation
\end{itemize}
\end{itemize}
\subsection{Decision procedure}
\label{sec:org035ac1d}
\begin{itemize}
\item What does it mean for a set to be decidable?
\begin{itemize}
\item If it has a decision procedure.
\begin{itemize}
\item Algorithm that gives a yes or no answer in a finite amount of time.
\end{itemize}
\end{itemize}
\item A set is \(X\) \uline{decidable} if there is a decision procedure for it.
\begin{itemize}
\item Characteristic function:
\begin{itemize}
\item 
\end{itemize}
\end{itemize}
\end{itemize}
\begin{equation*}
C_x(n) = \begin{cases} 1 & \text{if }n\in X
\\ 0 & \text{if }n\notin X
\end{cases}
\end{equation*}
Later we will see that decidable set \(\iff\) characteristic function computable.
\subsection{FS for addition}
\label{sec:org1ee8bd4}
Write down properties of addition to try and come up with a formal system to fit that interpretation.

Recursive definition of addition:
\begin{itemize}
\item \(x+1 = (x+1)\)
\item \(x+(n+1) = (x+n)+1\)
\begin{itemize}
\item This allows you to compute any addition
\item \(x+4 = (x+3)+1 = (x+2)+1+1 = (x+1) +1 + 1 + 1\)
\end{itemize}
\end{itemize}

Now translating it to the pq system:
\begin{center}
\begin{tabular}{lll}
Axiom & xp\mbox{-}qx\mbox{-} & x+1=(x+1)\\
\hline
IR & If xpyqz is a theorem & x+(n+1)=\\
 & then xpy\mbox{-}qz\mbox{-} & (x+n)+1\\
\end{tabular}
\end{center}

\subsection{FS for multiplication}
\label{sec:org0500e1e}
Want FS for multiplication (tq system)
x t y q z
\begin{itemize}
\item \(x \times 1 = x\)
\item \(x \times (n+1) = (x \times n) + x\)
\end{itemize}
Translating:
\begin{itemize}
\item Axiom: \(x t \mbox{-} q x\) (\(\aleph_0\) axioms)
\item Inference Rule: If \(x t y q z\) is a thm, then \(xt\overbrace{y\mbox{-}}^{n+1}qzx\)
\end{itemize}

\noindent\rule{\textwidth}{0.5pt}
Modification of tq system above.
\begin{itemize}
\item Alphabet: t q \mbox{-} C
\item Second Inference Rule: If \(x\mbox{-}ty\mbox{-}qz\) is a theorem then \(Cz\) is a theorem. (x,y,z non-empty strings of \mbox{-})
\item Example theorems:
\begin{itemize}
\item \mbox{-}t\mbox{-}q\mbox{-}
\item \mbox{-}\mbox{-}t\mbox{-}\mbox{-}q\mbox{-}\mbox{-}\mbox{-}\mbox{-}
\item C\mbox{-}\mbox{-}\mbox{-}\mbox{-}
\item \mbox{-}\mbox{-}\mbox{-}t\mbox{-}\mbox{-}q\mbox{-}\mbox{-}\mbox{-}\mbox{-}\mbox{-}\mbox{-}
\item C\mbox{-}\mbox{-}\mbox{-}\mbox{-}\mbox{-}\mbox{-}
\end{itemize}
\end{itemize}
Cx is true if \(x\) is a composite number (not a prime)
\begin{itemize}
\item If Cx is not a Cqt theorem, then Px (prime)
\begin{itemize}
\item Can we add this as another inference rule? No. Not saying how to get primes, just what isn't a prime. It's \uline{not an inference rule}, you have to be able to apply an inference rule mechanically (has to be decidable)
\end{itemize}
\end{itemize}
\subsection{Recursively enumerable set (r.e.)}
\label{sec:org973c6e8}
A recursively enumerable set can be generated as theorems of a formal system.
\begin{itemize}
\item Ex. Natural numbers
\end{itemize}
\subsection{Recursive set}
\label{sec:org3a48eac}
A set is recursive if it is r.e. and its complement is also r.e. Only want to talk about the complement in a clearly defined realm (universe).
\begin{itemize}
\item Well-formed expressions:
\end{itemize}
\begin{center}
\includegraphics[width=.9\linewidth]{./Images/i5.png}
\end{center}
\begin{itemize}
\item In GEB, he calls the circle the figure and the ground the non-thms that are the non-thms of the circle but they are theorems themselves.
\item Recursive sets are decidable. Why?
\item Can a set be recursively enumerable but not recursive.
\end{itemize}
\section{Lecture 8 \textit{<2017-09-28 Thu>}}
\label{sec:org65b3ea7}
\subsection{Theorems}
\label{sec:org87ceecc}
\(P\) is equivalent to \(Q\) relative to \(A_1\ldots A_n\):
\begin{itemize}
\item \(A_1 \ldots A_n, P\) prove \(Q\)
\item \(A_1 \ldots A_n, Q\) prove \(P\)
\end{itemize}
\subsubsection{E.g.}
\label{sec:org99e4d39}
Relative to Euclid's axioms:

Proclus' axiom
\begin{itemize}
\item If a line intersects 2 parallels it must intersect the other
\end{itemize}
Playfail's axiom
\begin{itemize}
\item If a line is parallel to a point, then there exists one parallel containing that point
\end{itemize}
Parallel postulate

Are equivalent.
\begin{itemize}
\item If you cannot prove \(P\) from \(A_1\ldots A_n\) then \(P\) is \uline{independent} of \(A_1 \ldots A_n\)
\end{itemize}

Parallel postulate is independent from the axioms of Euclid.
\begin{itemize}
\item One way to show:
\begin{itemize}
\item Give a model for \(A_1 \ldots A_n\) in which \(P\) is false. (If a model makes \(P\) false, then \(P\) cannot be a theorem.)
\end{itemize}
\item How to show that there are certain models for Euclid's axioms where the parallel postulate is false? Well, we have to come up with a model.
\end{itemize}

\begin{enumerate}
\item Playfail's axiom
\label{sec:org73dbe69}
There exists exactly one parallel to a given line through a given point.
\begin{itemize}
\item What would it mean for this to be false?
\begin{itemize}
\item Playfair's axiom can be false in \(2\) ways:

a) More than one parallel exists

b) No parallel exists
\end{itemize}
\end{itemize}

b)
\begin{itemize}
\item Line -> great circle on a sphere
\end{itemize}
\begin{center}
\includegraphics[width=.9\linewidth]{./Images/i6.png}
\end{center}
All great circles intersect, no parallels (Elliptic geometry)

\begin{itemize}
\item Point -> Point and it's antipode
\end{itemize}

a) 
\begin{itemize}
\item Line -> line inside disc
\item Point -> point inside the disk
\end{itemize}


Infinitely many parallel lines
(Hyperbolic geom.)

\begin{center}
\includegraphics[width=.9\linewidth]{./Images/i8.png}
\end{center}
\end{enumerate}

\subsection{PQ*-system}
\label{sec:org75dea9f}
\begin{itemize}
\item Ax. schema 1: xp\mbox{-}qx\mbox{-}
\item IR: xpyqz \(\to\) xpy-qz-
\item Ax. schema 2: xp \mbox{-} qz

\item Interpretation: 
\begin{itemize}
\item p -> plus

\item q -> equals

\item \mbox{-} -> unit
\end{itemize}

\item \mbox{-}p\mbox{-}q\mbox{-}\mbox{-}
\item Now, \mbox{-}\mbox{-}p\mbox{-}q\mbox{-}\mbox{-} is a thm
\item Meaning: 2+1 = 2 false.
\item Complete but not sound with respect to interpretation 1 (p -> plus)
\begin{itemize}
\item Different interpretation (2):

\begin{itemize}
\item p -> plus

\item q -> greater or equal

\item \mbox{-} -> unit

\item Sound but not complete with respect to interpretation 2.

\begin{itemize}
\item Ex. \mbox{-}\mbox{-}\mbox{-}p\mbox{-}\mbox{-}q\mbox{-} is not a theorem, but \(3+1 \geq 1\) is a truth

\item Axiom schema 2 only gives you things greater by 1
\end{itemize}
\end{itemize}

\item Different interpreation (3):

\begin{itemize}
\item p -> plus

\item q -> greater by 1 or equal

\item \mbox{-} -> unit

\item Sound and complete with respect to interpretation 3
\end{itemize}
\end{itemize}
\end{itemize}
\subsection{Propositional Logic}
\label{sec:org6aa4092}
\begin{itemize}
\item Today's presentation is harder for those who already know propositional logic, next week will be the standard presentation.
\end{itemize}
\subsubsection{Formula trees:}
\label{sec:org9362c0e}
\begin{itemize}
\item Language: Propositional variables: \(P_0, P_1, P_2, \ldots\)
\begin{itemize}
\item Unary connective: \(\sim\) (negation)
\item Binary connectives: \(\wedge\) (conjunction)
\begin{itemize}
\item \(\vee\) (disjunction)
\item \(\supset\) (implication)
\end{itemize}
\end{itemize}
\end{itemize}
\begin{enumerate}
\item Inductive Definition
\label{sec:org59f3dd8}
\begin{enumerate}
\item Base clause: A prop. variable is a formula tree
\item Inductive clauses: If \(A,B\) are formula trees then \begin{center}
\includegraphics[width=.9\linewidth]{./Images/i9.png}
\end{center} are also formula trees (with A, B as subtrees)
\item Nothing else is a formula tree
\end{enumerate}
\item E.g.
\label{sec:org4b7fa1b}
\begin{center}
\includegraphics[width=.9\linewidth]{./Images/i10.png}
\end{center}
The tree to the right has 5 subtrees (main connective is not a subtree of itself)
\item Truth value assignment
\label{sec:orgf5d1a72}
A \uline{truth value assignment} is a function from propositional variables to \(\{T, F\}\) (True, False). The \uline{truth value} of a formula tree A under the truth value assignment f is:
\begin{itemize}
\item Case 1: A is a propositional variable: f(A)
\begin{itemize}
\item E.g. \(f(P_0)=T\), \(f(P_1)=F\), \(f(P_{27})=T\), \(f(P_{5})=F\)
\end{itemize}
\item Case 2: A is of the form: \begin{center}
\includegraphics[width=.9\linewidth]{./Images/i11.png}
\end{center}
\begin{itemize}
\item Truth values:
\end{itemize}
\end{itemize}
\begin{center}
\begin{tabular}{ll}
B & A\\
\hline
T & F\\
F & T\\
\end{tabular}
\end{center}
\begin{itemize}
\item Case 3: A is of the form \begin{center}
\includegraphics[width=.9\linewidth]{./Images/i12.png}
\end{center}
\end{itemize}

\begin{center}
\begin{tabular}{lllll}
B & C & A\(_{\text{1}}\) & A\(_{\text{2}}\) & A\(_{\text{3}}\)\\
\hline
T & T & T & T & T\\
T & F & F & T & F\\
F & T & F & F & T\\
F & F & F & F & T\\
\end{tabular}
\end{center}
\end{enumerate}
\section{Lecture 9 \textit{<2017-10-03 Tue>}}
\label{sec:org1155244}
\subsection{Propositional Logic}
\label{sec:org9efcf4f}
Most of this is on handout 2b.

Let's define logic as a formal system.
\begin{itemize}
\item Alphabet: \(P_0, P_1, P_2, \ldots\) (\(\aleph_0\) propositional variables)
\item Connectives: \(\wedge\) (conjunction), \(\vee\) (disjunction), \(\supset\) (implication), \(\sim\) (negation)
\item Parentheses
\end{itemize}

\subsubsection{Well-formed formulas (wff)}
\label{sec:orgc5d92ff}
\begin{enumerate}
\item Base clause: \(P_i\) is a wff (\(i \in \mathbb{N}\)) (called atomic)
\item Inductive clause: If \(A\) and \(B\) are wffs, then so are:
\begin{itemize}
\item \(\sim A\) "not"
\item \((A \wedge B)\) "and"
\item \((A \vee B)\) "or"
\item \((A \supset B)\) "implies"
\end{itemize}
\item Nothing else is.
\end{enumerate}

E.g. 
\begin{itemize}
\item \(P_0\supset P_1\) is \textbf{not well formed}, lack of parentheses.
\item \((P_0\supset P_1)\)
\item \((P_{27})\) is \textbf{not well formed}, shouldn't have parentheses in atomic form.
\item \((\sim P_1 \wedge \sim (P_0 \supset P_2))\)
\item Can be shown as: \begin{center}
\includegraphics[width=.9\linewidth]{./Images/i13.png}
\end{center}
\end{itemize}
Convention: outer parens are omitted (except if asked for a well formed expression explicitly, as parentheses are required to comply with rules that give us the nice structure)
\begin{enumerate}
\item Interpretation
\label{sec:org8ff4385}
Propositional variables \(\to\) truth values: 
\begin{center}
\begin{tabular}{ll}
True & False\\
\hline
T & F\\
1 & 0\\
T & \(\perp\)\\
\end{tabular}
\end{center}

(Bivalence)

\item Truth tables
\label{sec:org65ea86a}
A,B (metavariables that stand for wff):
\begin{center}
\begin{tabular}{lllll}
A B & A \(\wedge\) B & A \(\vee\) B & \(\sim\) A & A \(\supset\) B\\
\hline
T T & T & T & F & T\\
T F & F & T & F & F\\
F T & F & T & T & T\\
F F & F & F & T & T\\
\end{tabular}
\end{center}
\begin{itemize}
\item \(A \supset B\) -> if \ldots{} then \ldots{}
\begin{itemize}
\item A is the antecedent
\item B is the consequent
\item This is material implication, not causal implication
\item The light (B) can be on even if I didn't flip the switch (A)
\end{itemize}
\item Ex.
\end{itemize}
\begin{center}
\includegraphics[width=.9\linewidth]{./Images/i14.png}
\end{center}

\begin{center}
\begin{tabular}{llll}
\(P_1\) & \(\wedge\) & \(\sim\) & \(P_1\)\\
\hline
T & F & F & T\\
F & F & T & F\\
\end{tabular}
\end{center}


\begin{center}
\begin{tabular}{llll}
\(P_1\) & \(\vee\) & \(\sim\) & \(P_1\)\\
\hline
T & T & F & T\\
F & T & T & F\\
\end{tabular}
\end{center}
Something that is always true is a \textbf{tautology}.
\begin{itemize}
\item Two wff are \uline{logically equivalent} if their TV agrees on all possible TV-assignments:
\end{itemize}
\begin{center}
\begin{tabular}{lll}
A B & A \(\supset\) B & \(\sim\) A \(\vee\) B\\
\hline
T T & T & T\\
T F & F & F\\
F T & T & T\\
F F & T & T\\
\end{tabular}
\end{center}

Do we read \(\sim A \vee B\) as \((\sim A) \vee B\) or \(\sim(A \vee B)\)?
\begin{itemize}
\item \((\sim A)\vee B\) due to the way we defined wff
\end{itemize}

Minimal sets of connectives: 
\begin{itemize}
\item \(\{\sim, \vee \}\)
\item \(\{\sim, \wedge\}\)
\item \(\{\sim, \supset\}\)
\item \(\implies\) Sheffer-Stroke
\end{itemize}

For a wff with \(n\) prop. vars, the truth table has \(2^n\) lines.

Is finding out if a proposition is a tautology decidable or not? Yes, just write out the truth table.
\item Inferences
\label{sec:org2ff07d1}
An inference is \textbf{valid} if it is impossible for all the premises to be true and the conclusion false at the same time.

\begin{center}
\begin{tabular}{lll}
 & Premises & Conclusion\\
\hline
A B & A A \(\supset\) B & B\\
T T & T T & T\\
T F & T F & F\\
F T & F T & T\\
F F & F T & F\\
\end{tabular}
\end{center}
This is a valid inference, when both premises are true, the conclusion is also true. Thus:
\begin{itemize}
\item A, A \(\supset\) B \(\models\) B
\begin{itemize}
\item Where \(\models\) is the (semantic) consequence
\item Can check if something is semantically implied by checking the truth table and when all premises are true.
\end{itemize}
\end{itemize}
Generalizing: \(A_1, \ldots A_n \models B\)
\begin{itemize}
\item \(\models\) B (tautology)
\end{itemize}
\item Natural Deduction (Syntax)
\label{sec:org0556f1b}
Introduced by Gentzen, 1934.

\(\frac{\text{Premises}}{\text{Conclusion}}\)
\begin{itemize}
\item \(\frac{A \ A \supset B}{B} \supset\) Elimination (Implication Elimination) or MODUS PONENS
\item \(\frac{A \ B}{A \wedge B} \wedge\) Introduction (since it introduces conjunction)
\item \(\frac{A\wedge B}{A}\wedge\) Elim
\item \(\frac{A \wedge B}{B}\wedge\) Elim
\begin{itemize}
\item Not the same as the rule above! You cannot get to \(B\) from the first one, you must use this one.
\item Also, \(A \wedge B\) and \(B \wedge A\) are not the same! They might have the same meaning, but they are different as strings, syntactically
\end{itemize}
\item \(\frac{A}{A \vee B}\vee\) Intro
\item \(\frac{A}{B \vee A}\vee\) Intro
\end{itemize}
Missing: 
\begin{itemize}
\item \(\vee\) Elim
\item \(\supset\) Intro
\item \(\sim\) Intro
\item \(\sim\) Elim
\end{itemize}
\end{enumerate}
\section{Lecture 10 \textit{<2017-10-05 Thu>}}
\label{sec:orga9ac39c}
\subsection{Natural Deduction}
\label{sec:org965a7a8}
\begin{itemize}
\item Proof system for propositional logic
\item In the land of syntax when we do this
\begin{itemize}
\item Remember that if something looks different, it is different
\item \(A \wedge B\) is not the same as \(B \wedge A\)!
\end{itemize}
\end{itemize}

\noindent\rule{\textwidth}{0.5pt}
Some rules:
\begin{itemize}
\item \(\frac{A \ A\supset B}{B}\supset Elim\)
\item \(\frac{A \ B}{A\vee B}\vee Intro\)
\item \(\frac{A \wedge B}{A}\wedge ElimR\)
\item \(\frac{A \wedge B}{B}\wedge ElimL\)
\item \(\frac{A}{A \vee B}\vee Intro\)
\item \(\frac{A}{B \vee A}\vee Intro\)
\end{itemize}
Note that \(A, A \supset B \models B\) consists of semantics, not \textbf{syntax}. The above rules mentioned are rules to infer other things syntactically.

\(\frac{\frac{A \wedge B}{B}\wedge ElimL \ \frac{A \wedge B}{A}\wedge ElimR}{B \wedge A}\wedge Intro\) can be abbreviated as:
\(A \wedge B \vdash_{ND} B \wedge A\) (ND stands for natural deduction, don't confuse this symbol with the semantic one)
\begin{itemize}
\item If you can get from \(A\) to \(B\), then you can box A (canceling this assumption \(A\)) with a subscript of the amount of steps.
\end{itemize}
\begin{center}
\includegraphics[width=.9\linewidth]{./Images/i15.png}
\end{center}
\begin{itemize}
\item \(\underbrace{\frac{[A]_2 \ [B]_1}{\frac{B \supset A}{A \supset (B \supset A)}\supset Intro_2} \supset Intro_1}_{\vdash_{ND} A \supset (B \supset A)}\)
\begin{itemize}
\item A and B don't prefix \(\vdash\) since they were eliminated (boxed)
\item You can get final result from no assumptions
\end{itemize}
\item \(\underbrace{\frac{A \ [B]_1}{B \supset A}\supset Intro_1}_{A \vdash_{ND} B \supset A}\)
\url{./Images/i16}
\item A and B both have subscript 1 because they're eliminated at the same time
\item This is like a formal definition for a proof by cases
\end{itemize}
\subsection{Exercise}
\label{sec:orge2224a7}
Prove: \(A\vee C, A \supset B, C \supset D \vdash_{ND} B \vee D\)
\url{./Images/i17}

\noindent\rule{\textwidth}{0.5pt}
\begin{itemize}
\item New symbol: \(\perp\) false/falsum
\item Generated by:
\begin{itemize}
\item \(\frac{A \ \sim A}{\perp}\sim Elim\)
\end{itemize}
\item Can use to:
\begin{itemize}
\item \(\frac{\perp}{A}Ex falsum\) (can introduce anything)
\end{itemize}
\item \begin{center}
\includegraphics[width=.9\linewidth]{./Images/i18.png}
\end{center}
RAA is a proof by contradiction.
\end{itemize}
Prove: \(\vdash_{ND}\sim A \supset (A \supset B)\)
\end{document}