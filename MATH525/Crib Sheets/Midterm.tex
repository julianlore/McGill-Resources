\documentclass[10pt,landscape]{article}
\usepackage{multicol}
\usepackage{calc}
\usepackage{ifthen}
\usepackage[landscape]{geometry}
\usepackage{hyperref, amsmath,tabularx, graphicx, pdfpages, blkarray}
\usepackage{xcolor}
\allowdisplaybreaks

% To make this come out properly in landscape mode, do one of the following
% 1.
%  pdflatex latexsheet.tex
%
% 2.
%  latex latexsheet.tex
%  dvips -P pdf  -t landscape latexsheet.dvi
%  ps2pdf latexsheet.ps


% If you're reading this, be prepared for confusion.  Making this was
% a learning experience for me, and it shows.  Much of the placement
% was hacked in; if you make it better, let me know...


% 2008-04
% Changed page margin code to use the geometry package. Also added code for
% conditional page margins, depending on paper size. Thanks to Uwe Ziegenhagen
% for the suggestions.

% 2006-08
% Made changes based on suggestions from Gene Cooperman. <gene at ccs.neu.edu>


% To Do:
% \listoffigures \listoftables
% \setcounter{secnumdepth}{0}


% This sets page margins to .5 inch if using letter paper, and to 1cm
% if using A4 paper. (This probably isn't strictly necessary.)
% If using another size paper, use default 1cm margins.
\ifthenelse{\lengthtest { \paperwidth = 11in}}
	{ \geometry{top=.4in,left=.4in,right=.4in,bottom=.4in} }
	{\ifthenelse{ \lengthtest{ \paperwidth = 297mm}}
		{\geometry{top=1cm,left=1cm,right=1cm,bottom=1cm} }
		{\geometry{top=1cm,left=1cm,right=1cm,bottom=1cm} }
	}

% Turn off header and footer
\pagestyle{empty}
 

% Redefine section commands to use less space
\makeatletter
\renewcommand{\section}{\@startsection{section}{1}{0mm}%
                                {-1ex plus -.5ex minus -.2ex}%
                                {0.5ex plus .2ex}%x
                                {\normalfont\large\bfseries}}
\renewcommand{\subsection}{\@startsection{subsection}{2}{0mm}%
                                {-1explus -.5ex minus -.2ex}%
                                {0.5ex plus .2ex}%
                                {\normalfont\normalsize\bfseries}}
\renewcommand{\subsubsection}{\@startsection{subsubsection}{3}{0mm}%
                                {-1ex plus -.5ex minus -.2ex}%
                                {1ex plus .2ex}%
                                {\normalfont\small\bfseries}}
\makeatother

% Define BibTeX command
\def\BibTeX{{\rm B\kern-.05em{\sc i\kern-.025em b}\kern-.08em
    T\kern-.1667em\lower.7ex\hbox{E}\kern-.125emX}}

% Don't print section numbers
\setcounter{secnumdepth}{0}


\setlength{\parindent}{0pt}
\setlength{\parskip}{0pt plus 0.5ex}


% -----------------------------------------------------------------------

\begin{document}
\raggedright
\footnotesize
\begin{multicols*}{3}
% multicol parameters
% These lengths are set only within the two main columns
%\setlength{\columnseprule}{0.25pt}
\setlength{\premulticols}{1pt}
\setlength{\postmulticols}{1pt}
\setlength{\multicolsep}{1pt}
\setlength{\columnsep}{2pt}

\begin{center}
  \Large{\textbf{MATH525 Crib Sheet}} \\
  Julian Lore
\end{center}
\begin{flalign*}
  V(aX + bY) & = a^2 V(X) + b^2 V(Y) + 2ab Cov(X,Y)
  \\ V(aX - bY) & = a^2 V(X) + b^2 V(Y) - 2ab Cov(X,Y)
  \\ X \perp Y & \implies Cov(X,Y) = 0
  \\ V(Y) & = E(Y^2) - (E(Y))^2 = E[(Y - E[Y])^2]
  \\ Cov(X,Y) & = \frac{1}{n} \sum_{i=1}^n (x_i - E(X))(y_i - E(Y))
  \\ Cov(X,Y) & = E \left[\left(X - E[X]\right)\left(Y - E[Y]\right)\right]
  \\ Cov(X,Y) & = E(XY) - E(X)E(Y)
  \\ V(X) & = Cov(X,X)
  \\ Bias(\hat{\theta}) & = E[\hat{\theta}] - \theta
\end{flalign*}
\section{Key Definitions}
\textbf{census} - measuring the quantity of interest
\underline{exactly}
\\ \textbf{observation unit/population unit} - single member of the
population
\\ \textbf{target population} - set of observation units that we want
to estimate the quantity for
\\ \textbf{sample} - subset of population units that we will measure
\\ \textbf{sample population} - set of population units who could ever
be sampled
\\ \textbf{sampling unit} - unit that can be selected for a sample
\\ \textbf{sampling frame} - list of all sampling units in the sample population
\\ We hope that everyone in sample pop belongs to target pop, but not
always the case. Perfect world would be target pop = sample pop.
\\ \textbf{selection bias}: part of target pop is not in sampled
pop. \textbf{judgment sample}: deliberately or purposely selecting
representative sample, sample units that you judge are
representative. \textbf{undercoverage}: failing to include all of
target pop in sampling frame. \textbf{overcoverage}: including pop
units in sampling frame that are not in target pop.
\\ \textbf{cluster sample}: randomly sample strata and take SRS or
census within each (can't sample all strata)
\\ \textbf{systematic sample}: choose a random starting point and then
take every $k$-th unit in the list
\\ \textbf{Probability sampling}: Each unit has a \underline{known}
prob of being sampled.
\section{Simple Random Sample} Select one of all possible subsets of
$n$ population units (this is without replacement, with replacement
can also be done). $\binom{N}{n}$ possible samples, each equally
likely. So $P(s) = \frac{1}{\binom{N}{n}}$ and $\pi_i = \frac{n}{N}$
(probability of including unit $i$).
\subsection{Estimation}
\begin{flalign*}
  t & = \sum_{i=1}^N y_i = N \overline{y}_U &
\\ \hat{t}_s & = \dfrac{N
  \sum_{s\in S} y_s}{n} = N \overline{y}_S (\textbf{unbiased})
\\ \overline{y}_u & = \frac{\sum_{i=1}^N y_i}{N}
= \frac{t}{n} 
\\ \overline{y}_S & =
\frac{\sum_{s\in S}y_s}{n} = \frac{\hat{t}_s}{N}(\textbf{unbiased})
\intertext{Under SRS:}
E(\hat{t}_s)
& =t, E(\overline{y}_S) = \overline{y}_U
\\ V(\overline{y}_s) & = \left(1 - \frac{n}{N}\right) \frac{S^2}{n}
\implies SE(\overline{y}_s) = \sqrt{ \left(1 - \frac{n}{N}\right)\frac{S^2}{n}}
    \\ \hat{SE}(\overline{y}_S) & =
\sqrt{\frac{s^2}{n} \left(1 - \frac{n}{N}\right)}
\\ V(\hat{t}_s) &= N^2 \left(1 - \frac{n}{N}\right)
\frac{S^2}{n} \implies SE(\hat{t}_s) = N \sqrt{ \left(1 - \frac{n}{N}\right)\frac{S^2}{n}}
\intertext{where}
S^2 & = \frac{\sum_{i=1}^N (y_i -
  \overline{y}_u)^2}{N-1} = \frac{\sum_{i=1}^N y_i^2 - \frac{1}{N}
  \left(\sum_{i=1}^N y_i\right)^2}{N - 1}
\\ & = \frac{\sum_{i=1}^N y_i^2 - N
  \overline{y}_u^2}{N-1}
\end{flalign*}
\\Under SRS: . For fixed $N$, as $n \to N, V(\hat{t}_s) \to
0$.
\\Estimate $S^2$ with sample variance $s^2$:
\begin{flalign*}
s^2 & = \frac{\sum_{s
    \in S}(y_s - \overline{y}_s)^2}{n-1} = \frac{\sum_{s \in S} y_i^2
  - n \overline{y}_u}{n-1} &
\end{flalign*}
\\ \textbf{Confidence interval for $\overline{y}_U$} $N \overline{y}_S
\pm z_{\alpha/2} \hat{SE}(\overline{y}_S)$
\\ \textbf{Finite population correction (fpc)} $\left(1 -
  \frac{n}{N}\right)$
\\ \textbf{Hajek} $N_{\nu} - n_{\nu} \to \infty$ then $\hat{t}_S \sim
N\left(t, N^2 \left(1 - \frac{n}{N}\right) \frac{S^2}{n}\right),
\hat{t}_S \pm 1.96 \sqrt{N^2 \left(1 -
    \frac{n}{N}\right)\frac{s^2}{n}}$ ($95\%$ CI)
\subsection{Sample Size Estimation}
\textbf{Absolute error:} $Pr(\left|\overline{y}_s - \overline{y}_U
  \leq e\right|) = 1 - \alpha$, $e$ is called the \textbf{margin of error}
\\ \textbf{Relative error:} $Pr \left(\frac{\left|\overline{y}_s -
      \overline{y}_U\right|}{\left|\overline{y}_u\right|} \leq
  r\right) = 1 - \alpha$
$$n = \frac{S^2z^2_{\alpha/2}}{e^2 + \frac{S^2z^2_{\alpha/2}}{N}}
\text{ or }
n = \frac{z^2_{\alpha/2}S^2}{(r\overline{y}_U)^2 +
  \frac{z^2_{\alpha/2}S^2}{N}}$$
\\ Naive sample size calc with no fpc: $$n_0 = \frac{S^2
  z^2_{\alpha/2}}{e^2}
\implies n = \frac{n_0}{1+\frac{n_0}{N}}$$
\subsection{Weights}
\begin{flalign*}
  \hat{t}_S &= \sum_{i \in S} \frac{N}{n}y_i = \sum_{i \in S}w_i y_i
  \intertext{where}
  w_i & = \frac{N}{n} = \frac{1}{\pi_i} = \frac{1}{Pr(z_i=1)}
  \intertext{and $z_i$ is indicator function}
\overline{y}_S & = \frac{\hat{t}}{N} = \frac{\sum_{i\in
    S}w_iy_i}{\sum_{i \in S} w_i}
\end{flalign*}
All weights are same in SRS. A sample in which every unit has same
sampling weight is called a \textbf{self-weighting} sample
\subsection{Model-based perspective}
Assume model for $Y_1, \ldots, Y_n$, these are now
\underline{random} (instead of only $Z_i$ being random).
$Y_1, \ldots, Y_n \stackrel{iid}{\sim} f_Y(y |
\theta)$, i.e. $E_f(Y_j) = \mu, V_f(Y_j) = \sigma^2\ \forall j$
\\ Want to estimate $T = \sum_{i=1}^N Y_i$
\\ Need to \textbf{predict} values for $y_i$ not in the sample.
\\ Best linear unbiased predictor: $$\hat{T} = \sum_{i \in S} Y_i +
\sum_{i \notin S}\hat{Y}_i = \sum_{i \in S}Y_i + \frac{N-n}{n}\sum_{i
  \in S} Y_i = \frac{N}{n}\sum_{i \in S}Y_i$$
(because of common mean)
\\ This is model-unbiased (if mean and var not different,
i.e. assumptions made under model are correct):
\begin{flalign*}
E_f(\hat{T} - T) &=
\frac{N}{n}\sum_{i \in S}E_f(Y_i) - \sum_{i=1}^N E_f(Y_i)
=\frac{N}{n}(n\mu) - N\mu =0 &
\\ E_f[(\hat{T} - T)^2] &= N^2 \left(1 - \frac{n}{N}\right)
\frac{\sigma^2}{n}
\end{flalign*}
CLT applies to $\underline{Y}_s$ because of model assumptions
\\ $\overline{Y}_s \sim N \left(\mu, \frac{\sigma^2}{n}\right),
\hat{T} \pm z_{\alpha/2} \sqrt{ \left(1 - \frac{n}{N}
    \frac{\sigma^2}{n}\right)}$
\subsection{When to use SRS} \textcolor{red}{Do not} use if a
controlled experiment is better (i.e. is this brand of bath oil an
effective mosquito repellent), do not have a list of obs
units/expensive to take an SRS or have extra information to make a
more cost-effective scheme.
\\ \textcolor{green}{Good for} little extra info available or
interested in multivariate relationships and no need to take
stratified/cluster sample. Easier to perform.
\section{Stratified Sampling}
\begin{enumerate}
\item Divide pop units into $H$ subpops or \underline{strata}
  (requires additional info)
\item Take a probability sample \underline{within} each stratum
  (independently)
\item Make inference about target param within each stratum, then pool
  results together
\end{enumerate}
Strata should be \underline{disjoint and partition} the sampling frame.
\subsection{Stratified Random Sampling} SRS within each
stratum. Divide pop of $N$ sampling units into $H$ strata, $N_h$ units
in strata $h$. Membership of strata must be mutually exclusive. Must
know $N_1, \ldots, N_H$ s.t. $N = \sum_{h = 1}^H N_h$. Stratified with
equal strata size is not the same as SRS because you are forcing to
have samples in each strata.
\\ Independently take SRS of size $n_h$ from each stratum: $n =
\sum_{h=1}^{H}n_h$
\\ Population quantities:
\begin{flalign*}
  y_{hj} &= \text{ val of $j^{th}$ unit in stratum $h$} &
  \\t_h & =
  \sum_{j=1}^{N_h}y_{hj} = \text{ stratum $h$ total}
  \\ t  & = \sum_{h=1}^{H}t_h = \text{ pop tot}
  \\ \overline{y}_{hU} & = \frac{t_h}{N_h} =\text{ true stratum $h$
    mean}
  \\ \overline{y}_U & = \frac{t}{N} =\text{ true pop mean}
  \\S_h^2 & = \sum_{j=1}^{N_h} \frac{(y_{hj} -
    \overline{y}_{hU})^2}{N_h-1} = \text{ stratum $h$ pop var}
  \\ S^2 & = \frac{\sum_{h=1}^{H}\sum_{j=1}^{N_h}(y_{jh} - \overline{y}_U)^2}{N-1}
= \text{ pop var} 
\end{flalign*}
\\ Sample quantities:
\begin{flalign*}
\overline{y}_h & = \frac{\sum_{j \in
    S_h}y_{nj}}{n_h} &
\\ \hat{t}_h & = \frac{N_h}{n_h} \sum_{j \in S_n}
y_{hj} = N_h \overline{y}_h
\\ s_h^2 & = \sum_{j \in S_h} \frac{(y_{jh}
  - \overline{y}_U)^2}{n_h - 1}
\\ \hat{t}_{str} &= \sum_{h=1}^{H}\hat{t}_h = \sum_{h=1}^{H} N_h \overline{y}_h (\textbf{unbiased})
\\ V(\hat{t}_{str}) &= \sum_{h=1}^{H} N_h^2
\frac{s_h^2}{n_h} \left(1 - \frac{n_h}{N_h}\right)
\\ \overline{y}_{str} & = \frac{\hat{t}_{str}}{N} = \sum_{h=1}^{H}
\frac{N_h}{N} \overline{y}_h (\textbf{unbiased}) 
\\V(\overline{y}_{str}) &= \frac{1}{N^2} V(\hat{t}_{str})=\sum_{h = 1}^H \left(1 -
  \frac{n_h}{N_h}\right) \left(\frac{N_h}{N}\right)^2
\frac{s_h^2}{n_h}
\\ \overline{y}_{str}& \pm z_{\alpha/2} SE(\overline{y}_{str})
\end{flalign*}
\subsection{Weights}
\begin{flalign*}
\hat{t}_{str}& = \sum_{h=1}^{H} N_h \overline{y}_h = \sum_{h=1}^{H}
\sum_{j \in S_h} \frac{N_h}{n_h}y_{hj} = \sum_{h=1}^{H} \sum_{j\in
  S_h} w_{hj} y_{hj}
\intertext{where}
w_{hj} &= \frac{N_h}{n_h} = \frac{1}{Pr(Z_{hj} = 1)}
= \frac{1}{\pi_{hj}}
\intertext{because $\pi_{hj} = \frac{n_h}{N_h}$}
\overline{y}_{str} & = \frac{\hat{t}_{str}}{N} = \frac{\sum_{h=1}^H
  \sum_{j \in S_h}w_{hj}y_{hj}}{\sum_{h=1}^{H}\sum_{j \in S_h}
  w_{hj}}
\end{flalign*}
\subsection{Allocating Observations}
\textbf{Proportional allocation}: $$n_h \propto N_h \implies n_h =
\left(\frac{N_h}{N}\right)n \implies \pi_{hj} = \frac{n_h}{N_h} =
\frac{n}{N}$$
$\hat{t}_{str} = \frac{N}{n} \sum_{h=1}^{H} \sum_{j \in S_h}
y_{hj}$ (self-weighting sample)
\\ $(N-1) S^2 = \left[\sum_{h=1}^{H} (N_h - 1)S_h^2\right] +
\sum_{h=1}^{H} N_h (\overline{y}_{hU} - \overline{y}_U) \implies TSS =
SSW + SSB$ (total sum of squares = sum of squares within + sum of
squares between)
\\ $V_{prop}(\hat{t}_{str}) = \left(1 - \frac{n}{N}\right) \frac{N}{n}
\sum_{h=1}^{H} N_h S_h^2 = \left(1 - \frac{n}{N}\right) \frac{N}{n}
(SSW + \sum_{h=1}^{H} S_h^2)$
\\ $V(\hat{t}_{SRS}) = \left(1 - \frac{n}{N}\right)N^2 \frac{S^2}{n} =
\frac{N}{N-1}V_{prop}(\hat{t}_{str}) + \frac{(N-h)N}{n(N-1)} (SSB -
\sum_{h=1}^{H} s_h^2)$
\\ So $\hat{t}_{SRS}$ will have larger variance than $\hat{t}_{str}$
unless $SSB < \sum_{h=1}^{H} S_h^2 \implies \sum_{h=1}^{H} N_h
(\overline{y}_h - \overline{y}_U)^2 < \sum_{h=1}^{H} S_h^2$
(variability between clusters is smaller than variability of each
strata). Don't want all strata to have same mean, or else variability
between will be small, making prop alloc worse. We want stratum means
to differ a lot s.t. sum of squares is large, variability within
strata is smaller.
\\ \textbf{Cost}: $C = c_0 + \sum_{h=1}^{H} c_h n_h$. Minimize
$V(\hat{y}_{str})$ subject to the constraint that $C <
C_{max}$. \textbf{Optimal allocation} $$n_h \propto
\frac{N_hS_h}{\sqrt{c_h}} \implies \frac{n_h}{n} =
\left(\dfrac{\dfrac{N_hS_h}{\sqrt{c_h}}}{\sum_{t = 1}^H \dfrac{N_t
      S_t}{\sqrt{c_t}}}\right)$$
If costs are equal across strata $c_1 = c_2 = \ldots = c_h$, then
$$n_h \propto N_h S_h \implies \frac{n_h}{n} =
\frac{N_hS_h}{\sum_{t=1}^H N_tS_t}$$ this is \textbf{Neyman allocation}. If
$S_h$ is known for all $h$, then Neyman beats prop alloc. With no max
cost, Neyman gives optimal alloc.
\subsection{Determining Sample Size}
$V(\overline{y}_{str}) = \sum_{h=1}^{H} \left(1 -
  \frac{n_h}{N_h}\right) \left(\frac{N_h}{N}\right)^2
\frac{s_h^2}{n_h} \leq \frac{1}{n} \sum_{h=1}^{H} \frac{n}{n_h}
\left(\frac{N_h}{N}\right)^2S_h^2 = \frac{\nu}{n}$, where $\nu =
\sum_{h=1}^{H}\left(\frac{n}{n_h}\right) \left(\frac{N_h}{N}\right)^2
S_h^2$.
\\ So $\overline{y}_{str} \pm z_{\alpha/2} \sqrt{\nu/n}$
\\ $n = z^2_{\alpha/2}\nu/e^2$ for margin of error $e$
\subsection{Model-Based}
$Y_{hj} = \mu_h + \varepsilon_{hj}$ where $\varepsilon_{hj} \sim
f(\varepsilon), E_f(\varepsilon_{hj}) = 0, V_f(\varepsilon_{hj}) =
\sigma_h^2, \varepsilon_{hj} \perp \varepsilon_{hi}$ for $i \neq j$
and $\varepsilon_{hj} \perp \varepsilon_{ki}$ for $h \neq k$.
\begin{flalign*}
  T_h & = \sum_{j=1}^{N_h}Y_{hj}, T = \sum_{h=1}^{H} T_h &
  \\ \hat{T}_h &=
  \frac{N_h}{n_h} \sum_{j \in S_h} Y_{hj}, T = \sum_{h=1}^{H} \hat{T}_h
  \\ E_f[\hat{T}_{h} - T_h] & = 0 \text{ (similar to SRS model-based)}
  \\ E_f[(\hat{T} - T)^2] & = E_f \left[ \left(\sum_{h=1}^H \hat{T}_h - T_h\right)^2\right]
  \\ &=E_f \left[\sum_{h=1}^H (\hat{T}_h - T_h)^2 + \sum_{h=1}^H \sum_{k \neq h} (\hat{T}_h - T_h)(\hat{T}_k - T_k)\right]
  \\ & =  E_f[\sum_{h=1}^H (\hat{T}_h - T_h)^2] = N_h^2 \left(1 - \frac{n_h}{N_h}\right)
\frac{\sigma_h^2}{n_h}
\end{flalign*}
\subsection{When not to Stratify} Expensive/impossible to identify the
strata, need to know $N_1, \ldots, N_H$, need to choose strata and
defend why it's good.
\section{Example Problems}
Gas stations, consider gas prices in November and December, want to
estimate pop diff in average gas prices between two months. Design 1:
SRS $n$ gas stations in November and then SRS $n$ gas stations in
December (independently). Good estimator is $\hat{\overline{d}} =
\overline{y}_{Dec} - \overline{y}_{Nov}$, sample avgs unbiased for
$\overline{y}_{u,Month}$. so difference is
unbiased. $V[\hat{\overline{d}}] = V(\overline{y}_{Dec}) +
V(\overline{y}_{Nov})$
\\ Design 2: SRS of $n$ gas stations in November and then same
stations in December. $\hat{\overline{d^*}} = \overline{y}_{Dec} -
\overline{y}_{Nov}$, unbiased because sample means are unbiased. For
var: $\hat{\overline{d^*}} = \frac{1}{n} \sum_{i \in S} (y_{i, Dec} -
y_{i, Nov}) = \frac{1}{n} \sum_{i \in S} d_i \implies
V[\hat{\overline{d^*}}] = \left(1 - \frac{n}{N}\right)\frac{S_d^2}{n}$
where $S_d^2 = \frac{1}{N-1} \sum_{i=1}^N (y_{i,Dec} - u_{i,Nov} -
[\overline{y}_{u,Dec} - \overline{y}_{u, Nov}])^2 = V(Y_{Dec}) +
V(Y_{Nov}) - 2Cov(Y_{Dec}, Y_{Nov})$. Design 2 is better if covariance
or correlation is positive (lower var).
\\ Given margin of error for SRS total and want to compute $n$,
convert margin of error to mean and then use formula for $n$ for mean.
\\ Cov: Design based: $Cov(\overline{y}_n, \overline{y}_m) = Cov
\left(\sum_{i \in S_1} \frac{y_i}{n}, \sum_{j\in S_2}
  \frac{y_j}{m}\right) = Cov \left(\sum_{i=1}^N Z_i \frac{y_i}{n},
  \sum_{j=1}^N Z'_j \frac{y_j}{m}\right) = \sum_{i=1}^N \sum_{j=1}^N
y_i y_j \frac{1}{n} \frac{1}{m} \underbrace{Cov (Z_i,
  Z_j')}_{E(Z_iZ_j') - E(Z_i)E(Z_j')}$. Use $E(Z_i Z_j') = Pr(Z_i = 1,
Z_j'=1) = Pr(Z_j' = 1 \mid Z_i = 1)Pr(Z_i = 1)$
\\ Model based: $Cov(\overline{y}_n, \overline{y}_m) = \sum_{i \in
  S_1} \sum_{j \in S_2} \frac{1}{n} \frac{1}{m} Cov (Y_i, Y_j) =
\sum_{i \in S_1} \sum_{j \in S_2} \frac{1}{n} \frac{1}{m}
Cov(\epsilon_i, \epsilon_j) = \sum_{i \in S_1} \sum_{j\in S_2}
\frac{1}{n} \frac{1}{m} \times 0 = 0$
\end{multicols*}
\end{document}
